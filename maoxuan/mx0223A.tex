
\title{中国革命与中国共产党}
\date{一九三九年十二月十五日}
\thanks{从这篇原文可以看到,毛选中这篇文章谈到中国封建社会的资本主义萌芽的那些段落,都是原文所没有的,是事后补充的。有一小段落承认中小地主可以在反对帝国主义和大地主的时候,保持中立或暂时参加斗争,在收入毛选的时候则删掉。原文谈到四个级阶联盟的时候,并没有强调无产阶级领导,但在收入毛选的时候就补加了无产阶级领导的话。此外各种含有政治意义的修改也不少。}
\maketitle



\section{第一章 中国社会}

\subsection{第一节 中华民族}

我们中国是世界上最大国家之一,他的领土超过了整个欧洲的面积。在这个广大领土之上,有广大的肥田沃地,给我们以衣食之源;有纵横全国的大小山脉、大小高原、平原,给我们生长了广大的森林,贮藏了丰富的矿产;有很多的江河湖泽,给我们以舟楫与灌溉之利;有很长的海岸线,给我们以交通海外各民族的方便。从很早的古代起,我们中华民族的祖先就劳动、生息、繁殖在这块广大土地之上。

现在中国的国境:在东北、西北和西境的一部与社会主义苏维埃共和国联盟接壤。西方的一部和西南方与印度、不丹、尼泊尔接壤。南方与暹逻、缅甸和安南接壤,并和台湾邻近。东方与日本邻近和朝鲜接壤。这个地理上的国际环境,给予中国革命造成了外部的有利条件和困难条件。有利的是:与苏联接壤,与欧美一切主要帝国主义国家隔离较远,在其周围的许多国家中大部都是殖民地半殖民地国家。困难的是:日本帝国主义利用其海、陆、空与中国接近的关系,时刻都在迫害着中国的生存和中国的革命。

我们中国现在拥有四万万五千万人口,差不多占了全世界人口的四份之一。在这四万万五千万人口中,十分之九为汉人。此外,还有回人、蒙人、藏人、满人、苗人、夷人、黎人等等许多少数民族,虽开化的程度不同,但他们都有了长久的历史。中国是一个由多数民族结合而成的拥有广大人口的国家。

中华民族的发展(主要是汉族的发展),和世界上别的大民族同样,曾经经过了若干万年平等而无阶级的原始共产主义社会的生活。而从原始共产主义社会崩溃、社会生活转入阶级生活那个时代开始,经过奴隶社会、封建社会,直到现在,已有了五千年之久。在中华民族主要是汉族的开化史上,有素称发达的农业和手工业,有许多伟大的思想家、科学家、发明家、政治家与军事家,有丰富的文化典籍,还在三千年前,中国就有了指南针的发明。还在一千七百年前,已经发明了造纸法。在一千二百年前,已经发明了刻版印刷。在八百年前,更发明了活字印刷。火药的应用,也远在欧人之前。所以中国是世界文明发达最早的国家之一,中国已有了五千年的文明史。

中华民族不但是以刻苦耐劳著称于世,同时又是酷爱自由富于革命传统的民族。以汉族的历史为例,证明中国人民是不能忍受黑暗势力的统治的,他们每次都用革命的手段达到推翻与改造这种统治的目的。在汉族的数千年的历史上,有过几百次的农民暴动,反抗地主贵族的黑暗统治;而每次朝代的更换,都是由于农民暴动的力量才能得到成功的。中华民族的各族人民对于外来民族的压迫都是不愿意的,都是要用反抗的手段解除这种压迫的。他们只赞成平等的联合,而不赞成互相压迫。在中华民族的几千年的历史中,产生了很多的民族英雄与革命领袖,产生了很多的革命军事家、政治家、文学家与思想家。所以中华民族又是一个有光荣革命传统和优秀历史遗产的民族。

\subsection{第二节 古代的封建社会}

中国虽是一个伟大的民族国家,虽是一个地广人众、历史悠久而又富于革命传统与优秀遗产的国家;可是中国由从脱离奴隶制度进到封建制度以后,就长期的停顿起来。这个封建制度,自周秦以来一直延续了三千多年。由于封建制度的延续,就使得中国的经济、政治、文化,都长期的陷在发展迟缓甚至停滞的状态中。三千年来的中国社会是一个封建的社会。

中国封建时代的经济制度和政治制度,是由以下的各个主要特点造成的:

一、自足自给的自然经济占主要地位。农民不但生产自己需要的农产品,而且生产自己需要的大部分手工业品。农业交付地主贵族的地租,也主要是地主们自己享用,不是为了交换。那时虽有交换的发展,但在整个经济中不起决定的作用。

二、封建的统治阶级——地主、贵族以至皇帝,他们拥有最大部分的土地,而在农民则很少土地,或完全没有土地。农民用自己的工具去耕种地主、贵族和皇室的土地,并将收获的四成、五成、六成甚至七成,奉献给地主、贵族、皇室们享乐,这种农民实际上还是农奴。

三、不但地主、贵族和皇室依靠剥削农民的地租过活,而且地主阶级的国家又强迫农民缴纳贡税并强迫农民从事无偿的劳役,去养活一大群的国家官吏及为了镇压农民之用的军队。

四、保护这种封建剥削制度的,便是地主阶级的封建国家。如果说周是诸侯割据称雄的封建国家,那么自秦始皇统一中国以后,就建立了专制主义的中央集权的封建国家,同时,在某种程度上仍旧保留着封建割据的状态。在封建国家中皇帝有至高无上的绝对的权力,在各地方设官职以掌兵、刑、钱、谷等事,并依靠地主绅士作为全部封建统治的基础。

中国历代的农民,就在这种封建的经济剥削和封建的政治压迫之下,过着贫穷困苦的奴隶式的生活。农民被束缚于封建制度之下,没有人身的自由,地主对农民有随意打骂甚至处死之权,农民是没有任何政治权利的。由于地主阶级这样残酷的剥削和压迫所造成的农民的极端穷苦和落后,就是中国社会几千年在经济上和社会生活上停滞不前的基本原因。

封建社会的主要矛盾,是农民阶级与地主阶级的矛盾。

而在这样的社会中,只有农民与手工业工人是创造财富与创造文化的基本的阶级。

地主阶级对于农民的残酷的经济剥削和政治压迫,曾经不能不在历史上掀起无数的农民暴动以反抗地主阶级的统治。从秦朝的陈胜、吴广、项羽、刘邦,汉朝的新市、平林、赤眉、黄巾、铜马,隋朝的李密、窦建德,唐朝的黄巢,宋朝的宋江、方腊,元期的朱元璋,明朝的李自成,直至清朝的太平天国,总共不下数百次,都是农民的反抗运动,都是农民的革命战争。中国历史上农民暴动与农民战争的规模之大,是世界历史上所没有的。只有这种农民暴动与农民战争,才是中国历史进化的真正动力。因为每次农民暴动与农民战争的结果,都打击了当时的封建统治,因而也就多少变动了社会的生产关系与多少推动了社会生产力的发展。只是由于当时还没有新的生产力与新的生产方式,没有新的阶级力量,没有先进的政党,因为这种农民战争与农民暴动得不到先进阶级与先进政党的领导如同现在的无产阶级与共产党能够正确领导农民暴动与农民战争。这样,就使当时的农民革命总是陷于失败,总是在革命中与革命后被地主贵族利用了去,当作他们改朝换代的工具。这样,就在每次农民革命斗争停息以后,虽然多少有些进步,但是封建的经济关系和封建的政治制度,基本上依然继续下去。

这种情况,直至近百年来,才发生新的变化。

\subsection{第三节 现代的殖民地、半殖民地、半封建社会}

中国过去三千多年的社会是封建社会,前面已经说明了。那么,中国现在的社会是否还是完全的封建社会呢?不是,中国已经变化了。自从一八四〇年的鸦片战争以来,中国已经一步一步的变成了一个半殖民地半封建的社会。自从一九三一年“九一八”事变日本帝国主义武装侵略中国以来,中国又变成了一个殖民地、半殖民地、半封建的社会。现在我们就来说明这种变化的过程。

如第二节所述,中国的封建社会继续了三千多年。直到十九世纪的中叶,由于外国资本主义的侵入,才使这个社会的内部发生了重大的根本的变化。

外国资本主义的侵入,曾经对中国的社会经济起了分解的作用。因为外国资本主义的侵入,一方面破坏了中国自给自足的自然经济,破坏了城市的手工业及农民的家庭手工业;又一方面则促进了中国城乡商品经济的发展。

这些情形,不仅对中国封建经济的基础起了解体的作用,同时又给中国资本主义生产的发生造成了某些客观的条件与可能。因为自然经济的破坏,给资本主义造成了商品的销售市场,而大量农民和手工业者的破产,又给资本主义造成了劳动力的购买市场。

事实上,由于外国资本主义的刺激与封建经济结构的某些破坏,还在十九世纪下半期,还在六十年前,就开始有一部分商人、地主和官僚投资于新式工业。到了同世纪末年和二十世纪初,到了四十年前,中国民族资本主义便开始了初步的发展。到了二十年前,即第一次帝国主义世界大战的时期,由于欧美帝国主义国家忙于战争,暂时放松了对于中国的压迫,中国的民族工业,主要是纺织业、面粉业和丝织业,曾经得到了进一步的发展。在这一时期中,中国的纺织业,由一九一五年的二十二个厂,增加到一九二二年的四十四个厂;面粉业由一九一六年的六十七个厂,增加到一百零七个厂;丝织业增加了六十三个厂。在这一时期中,中国的银行,也增加了一百零八个。

中国民族资本主义发生和发展的过程,就是中国资产阶级与无产阶级发生和发展的过程,如果一部分的商人、地主和官僚是中国资产阶级的前身,那么,一部分的农民和手工业工人就是中国无产阶级的前身了。中国的资产阶级与无产阶级,作为两个特殊的社会阶级来看,他们是新产生的,他们是中国历史上没有过的阶级,他们从封建社会脱胎而来,构成了新的社会阶级。他们是两个互相关联又互相对立的阶级,他们是中国旧社会(即封建社会)产出的双生子。但是中国无产阶级的发生与发展,不但是伴随中国民族资产阶级的发生与发展而来,而且是伴随帝国主义在中国直接经营企业而来,所以中国无产阶级较之中国资产阶级的年龄和资格更老些,所以他的社会力量和社会基础也更广大些。

可是,上面所述的这一资本主义发展的新变化,还只是帝国主义侵入中国以来所发生的变化的一个方面,还有与这个变化同时存在而阻碍这个变化的另一个方面,这就是帝国主义勾结中国封建残余压迫中国资本主义的发展。

帝国主义列强侵入中国的目的,决不是要把封建的中国变成资本主义的中国。帝国主义列强的目的与此相反,他们是要把中国变成他们的半殖民地与殖民地。

帝国主义列强为了这个目的,曾经对中国采用了并且还继续采用着如同下面所说的那些军事的、政治的、经济的以及文化的一切压迫手段,使中国一步一步的变成了半殖民地与殖民地:

一、用战争打败了中国之后,帝国主义国家便抢去了中国的许多属国与一部分领土。日本占领了朝鲜、台湾、琉球、澎湖群岛与旅顺,英国占领了缅甸、不丹、尼泊尔与香港,法国占领了安南,而蕞尔小国如葡萄牙也占领了我们的澳门。割地之外,又索去了巨大的赔款。这样就大大打击了中国这个庞大的封建帝国。

二、帝国主义根据条约有在中国驻扎海军与陆军之权,有领事裁判权,并把全中国划分为几个帝国主义的势力范围。

三、帝国主义根据条约控制了中国一切重要的通商口岸,并把许多通商口岸划出一部分土地作为他们直接管理的租界,他们控制了中国的海关与对外贸易,控制了中国的交通事业(海上的、陆上的、空中的与内河的)。因此他们便能够使中国的农业生产服从于帝国主义的需要。

四、帝国主义还在中国经营了许多轻工业和一部分重工业的企业,以便直接利用中国的原料与廉价的劳动力,并以此与中国的民族工业进行直接的竞争。

五、帝国主义经过借款给中国政府,并在中国开设银行,垄断了中国的金融财政。因此他们就不但在商品竞争上压倒了中国的民族资本主义,而且在金融上、财政上扼住了中国的咽喉。

六、帝国主义从中国的通商都市直至穷乡僻壤,造成一个买办的和商业高利贷的剥削网,造成了为帝国主义服务的买办阶级和商业高利贷阶级,以便利其剥削广大的中国农民。

七、于买办阶级之外,帝国主义还需要一个更大的社会力量,作为他们统治中国的支柱,这种社会力量就是中国的封建残余。他们“首先和以前社会构造的统治阶级——封建地主、商业与高利贷资产阶级结了联盟,以进攻占大多数的民众。帝国主义到处企图保持资本主义前期的榨取形式(尤其是在乡村)用作反动联盟生存的基础”。(共产国际六次大会殖民地与半殖民地运动大纲)“帝国主义及其所有财政和军事力量之在中国,就是拥护且推动那些封建残余及其全部军阀官僚的上层建筑物,使他欧化又使他成为守旧的力量”。(一九二七年斯大林在共产国际执委会的演说)

八、为了造成中国军阀混战与镇压中国人民的必要。帝国主义曾经供给中国政府以大量的军火与大批的军事顾问。

九、帝国主义在所有上述这些办法之外,对于麻醉中国人民的精神一方面,也不放松,这就是他们的文化政策。传教、办学校、办报纸与吸引留学生等,就是这个政策的实施,其目的在于造就服从他们的知识干部与愚弄广大的中国人民。

十、帝国主义用所有上述各种办法一步一步的把中国变成了半殖民地。这种局面,都是帝国主义在多次残酷战争之后所造成的。例如一八四〇年英国的鸦片战争,一八五七年英法联军的进攻北京,一八八四年的中法战争,一八九四年的中日战争,一九〇〇年的八国联军进攻北京。在这些战争之后,中国就沦为各主要帝国主义国家共同宰割和互相争夺的半殖民地,出现了上述半殖民地局面。而且一九三一年“九一八”以来,日本帝国主义的大举进攻,更使中国一大块土地沦为日本的殖民地。

上述这些情形,就是帝国主义侵入中国以后新的变化的又一方面,就是把一个封建的中国变为一个半封建半殖民地与殖民地的中国的血迹斑斑的图画。

由此可以明白,帝国主义侵略中国,有其促使中国封建社会解体的一方面,使中国发生了资本主义因素,起一个封建社会变成了半封建的社会;但同时,残酷地统治了中国,把一个独立的中国变成了殖民地与半殖民地的社会。

将这两个方面的情形综合起来说,我们这个殖民地、半殖民地、半封建的社会,有如下的几个特点:

一、封建时代的自足自给的自然经济是被破坏了;但是,无论在抗战的中国政府统治区域,无论在日本占领区域,封建剥削的根基——地主阶级对农民的封建剥削,不但依旧保持着,而且与买办资本和高利贷资本的剥削结合在一起,在中国的社会经济生活中,占着显然的优势。

二、民族资本主义有了某些发展,并在中国政治的、文化的生活中起了一定的作用;但是,他没有成为中国社会经济的主要形式,他的力量是很软弱的,他是对于外国帝国主义和国内封建残余都有联系的。尤其是“九一八”以来,民族工业的绝大部分被日本帝国主义所摧毁,所掠夺,更大大改变了中国的局面。

三、皇帝和贵族的专制主义是被推翻了;但代之而起的不是军阀官僚的统治,就是地主与大资产阶级联盟的专政。在沦陷区则是日本帝国主义及其傀儡的统治。

四、帝国主义不但操纵了中国的财政和经济的命脉,并且操纵了中国的政治和军事的力量,在沦陷区,则一切被日本帝国主义所独占。

五、由于中国是在许多帝国主义统治或半统治之下,由于中国实际上处于长期的不统一状态之中,又由于中国的土地广大,就使得中国的经济、政治与文化的发展,表现出极端的不平衡。

六、由于帝国主义和封建残余的双重压迫,特别是由于日本帝国主义的大举进攻,中国的广大人民尤其是农民,日益贫困以至破产,他们过着饥寒交迫与毫无政治权利的生活。中国人民的特殊的贫困与不自由,是世界各民族中所少有的。

殖民地、半殖民地、半封建的中国社会的特点就是这样。

决定这种情况的,主要是日本帝国主义与国际帝国主义的势力,是外国帝国主义与国内封建残余相结合的结果。

帝国主义与中华民族的矛盾,封建残余与人民大众的矛盾,这就是现时中国社会的主要矛盾。(当然还有别的矛盾,例如资产阶级与无产阶级的矛盾,统治阶级内部的矛盾等)而帝国主义与中华民族的矛盾乃是各种矛盾中的最主要的矛盾。这些矛盾的斗争及其尖锐化,就不能不造成日益发展的革命运动。伟大的近代与现代的中国革命,是在这些基本矛盾的基础之上发生与发展起来的。

\section{第二章 中国革命}

\subsection{第一节 百年来的革命运动}

帝国主义与中国封建残余相结合,把中国变为半殖民地与殖民地的过程,也就是中国人民反抗帝国主义及其走狗的过程。从鸦片战争、太平天国运动、中法战争、中日战争、戊戌政变、义和团运动、辛亥革命、五四运动、五三十运动、北伐战争、土地革命,直至现在的抗日战争,都表现了中国人民不甘屈服于帝国主义及其走狗的不断的反抗精神。

中国人民,百年以来,不屈不挠再接再厉的英勇斗争,使得帝国主义至今不能灭亡全中国,也永远不能灭亡全中国。

现在日本帝国主义虽然竭其全力大举进攻中国,虽然有许多地主与大资产阶级分子,例如公开的汪精卫与暗藏的汪精卫之流,已经投降敌人或准备投降敌人,但英勇的中国人民不但奋战了三年之久,而且必然还要奋战下去,不到驱逐日本帝国主义出中国,使中国得到了完全的解放,是决不会停止的。

中国人民的民族革命斗争,从一八四〇年的鸦片战争算起,已经有了整整一百年的历史了,从一九一一年的辛亥革命算起,也有了三十年的历史了。这个革命的过程,现在还未完结,革命的任务还没有显著的成就,还要求全国人民,首先是中国共产党,担负起坚决奋斗的责任。

那么,这个革命的对象究竟是谁?这个革命的任务究竟是甚么呢?这个革命的动力是什么?这个革命的性质是什么?这个革命的前途又是什么呢?这些就是我们下面要来说明的。

\subsection{第二节 中国革命的对象}

依照第一章第三节的分析,我们已经知道了:中国现时的社会,是一个殖民地、半殖民地,半封建性质的社会。只有认清中国社会的性质,才能认清中国革命的对象、中国革命的任务、中国革命的动力、中国革命的性质、中国革命的前途转变。所以,认清中国社会的性质,就是说,认清中国的“国情”,乃是认清一切革命问题的基本根据。

中国现时社会的性质,既然是殖民地、半殖民地、半封建的性质,那么,中国现阶段革命的主要对象或主要敌人,究竟是谁呢?

不是别的,就是帝国主义与半封建势力,就是外国的资产阶级与本国的地主阶级。因为在现阶段的中国社会中,压迫和阻止中国社会向前发展的主要的东西不是别的,正是他们二者,二者互相勾结以压迫中国人民,而以帝国主义的民族压迫为最大的压迫,因而帝国主义是中国人民的第一个和最凶恶的敌人。

在日本武力侵入中国以来,中国革命的主要敌人是日本帝国主义与勾结日本公开投降或准备投降的一切汉奸。

中国资产阶级本来也是受着帝国主义压迫的,他们也曾经领导过光荣的革命斗争,也曾经在革命中起过主要的或部分的领导作用,例如:辛亥革命、北伐战争与当前的抗日战争。但是,他们曾经在一九二七至一九三六年这一个长时期内勾结帝国主义,并与地主阶级结成反动同盟,背叛了曾经援助过他们的朋友——共产党、无产阶级、农民阶级与其他小资产阶级,背叛了中国革命,变成了人民的公敌,造成了革命的失败。所以当时革命的人民与革命的政党(共产党),曾经不得不把资产阶级也当作革命对象之一。在抗日战争中,大地主大资产阶级的一部分,以汪精卫为代表,已经叛变,已经变成汉奸,所以抗日的人民也已经不得不把这些背叛民族利益的大资产阶级分子当作革命对象之一。

由此也可以明白,中国革命的敌人是异常强大的。中国革命的敌人不但有强大的帝国主义,而且有强大的半封建势力,而且在一定时期内还有勾结帝国主义与半封建势力以与人民为敌的资产阶级,因此,那种轻视中国革命敌人力量的观点是不正确的。

在这样的敌人面前,中国革命的长期性与残酷性就发生了。因为我们的敌人是异常强大的,革命力量就非在长期间内不能聚积与锻炼成为一个足以最后战胜敌人的力量。因为敌人对中国革命的镇压是异常残酷的,革命力量就非磨炼与发挥自己的顽强性,不能坚持自已的阵地与夺取敌人的阵地。因此,那种以为中国革命力量瞬间就可以组成,中国革命斗争顷刻就可以胜利的观点是不正确的。

在这样的敌人面前,中国革命的方法,中国革命的主要形式,不能是和平的,而必须是武装的,也就决定了。因为我们的敌人不给中国人民以和平活动的可能,中国人民没有任何的政治自由。斯大林说:“中国革命的特点是武装的人民反对武装的反革命”,这是异常正确的规定。因此,那种轻视武装斗争,轻视革命战争,轻视游击战争,轻视军队工作的观点,是不正确的。

在这样的敌人面前,革命的特殊根据地问题也就发生了。因为强大的帝国主义及其在中国的反动同盟军,总是长期地占据着中国的中心城市,如果革命队伍不愿意和帝国主义及其走狗妥协,而要坚持奋斗下去,如果革命队伍要准备蓄积和锻炼自己的力量,并避免与强大敌人在力量不够时作决定胜负的战斗,那就必须把落后的农村造成先进的巩固的根据地,造成军事上、政治上、经济上、文化上的伟大革命阵地,借以反对利用城市进攻农村区域的凶恶敌人,借以在长期战斗中逐渐争取革命的全部胜利。在这种情形下面,由于中国经济发展的不平衡(农村经济不完全依赖城市),由于中国土地的广大(革命势力有回旋的余地),由于中国反革命营垒内部不统一和充满着各种矛盾,由于中国革命主力军的农民的斗争是在共产党的领导之下,这样,就使在一方面,中国革命有在农村区域首先胜利的可能;而在另一方面,则又造成了革命的不平衡状态,而使争取革命全部胜利的事业带来了长期性和艰苦性。由此也就可以明白,在这种特殊根据地上进行的长期革命斗争,主要的是在中国共产党领导之下的农民游击战争。因此,忽视以农村区城作革命根据地的观点,忽视对农民进行艰苦工作的观点,忽视游击战争的观点,都是不正确的。

但是着重武装斗争,不是说可以放弃其他形式的斗争;相反,没有武装斗争以外的各种形式的斗争相配合,武装斗争就不能胜利。着重农村根据地上的工作,不是说可以放弃城市工作及尚未成为根据地的其他广大农村中的工作;相反,没有城市工作及其他农村工作,革命根据地就处于孤立,革命就会失败。而且革命的最后目的,是夺取作为敌人主要根据地的城市,没有必要的足够的城市工作,就不能达此目的。

由此也就可以明白,为要使革命在农村与城市都胜利,不破坏敌人斗争的主要工具即敌人的军队,也是不可能的。因此,瓦解敌军的工作就成为极端重要的工作。

由此也就可以明白,在敌人长期占领的反动的黑暗的城市与反动的黑暗的农村中进行共产党的宣传工作与组织工作,不能采取急性病的冒险主义的方针,必须采取长期埋伏积蓄力量以待时机的方针。其领导人民对敌斗争的策略,必须利用一切可以利用的公开合法的法律命令及社会习惯所许可的范围,从有理、有利、有节的观点出发,一步一步与稳打稳扎的去进行,决不是大唤大叫与横冲直撞的办法所能成功的。

\subsection{第三节 中国革命的任务}

既然现阶段上中国革命的敌人主要是帝国主义与封建残余。那么,现阶段上中国革命的任务是什么呢?

毫无疑义,主要的就是打击这两个敌人,就是对外推翻帝国主义压迫的民族革命,对内推翻封建残余压迫的民主革命,而首先的任务便是打击帝国主义的民族革命。

中国革命的两大任务,是互相关联的。如果不推翻帝国主义的统治,就不能消灭封建残余,因为帝国主义是封建残余的主要支持者。反之,如果不肃清封建残余,也不能推翻帝国主义的统治,因为封建残余是帝国主义统治中国的主要社会基础。所以,民族革命与民主革命这样两个基本任务,是互相分别,又是互相统一的。

中国今天的民族革命任务,主要是反对侵入国土的日本帝国主义,而民主革命任务,则是在抗日战争中为了争取胜利的必要而去完成的,两个革命任务已经联系在一起了,那种把民族革命与民主革命分为截然对立的两个革命阶段的观点,已经是不合时宜的了。

\subsection{第四节 中国革命的动力}

根据现阶段中国社会性质、中国革命对象、中国革命任务的分析与规定,中国革命的动力是什么呢?

既然中国社会是一个殖民地、半殖民地、半封建的社会,既然中国革命所反对的对象主要的是外国帝国主义在中国的统治和内部的半封建势力,既然中国革命的任务是在推翻这两个压迫者的民族革命与民主革命,那么,在中国社会的各个阶级各个阶层中,有些什么阶级有些什么阶层可以充当反对帝国主义与反对封建势力的力量呢?这就是现阶段上中国革命的动力问题,认清这个革命动力问题,才能正确的解决中国革命的基本策略问题。

现阶段的中国社会里,有些什么阶级呢?有地主阶级、有资产阶级,这两个阶级都是上层统治阶级。又有无产阶级、有农民阶级、有各种类型的小资产阶级,这后面三个阶级,在今天的最广大领土上,还是被统治阶级。

所有这些阶级,他们对中国革命的态度和立场如何,全依他们在社会经济中所占的地位来决定。所以,社会经济的性质,不仅规定了革命的对象与任务,又规定了革命的动力。

我们现在就来分析一下中国社会的各阶级。

\subsection{一、地主阶级}

地主阶级是封建残余的代表,是帝国主义统治中国的主要社会基础,是剥削农民和压迫农民的阶级,是在政治上、经济上、文化上阻碍中国社会前进而没有丝毫利益的阶级。

因此,作为阶级来说,地主阶级是革命的对象,而不是革命的动力。

但是地主阶级中,最反动的是大地主阶层。至于中小地主,特别是破产与半破产的小地主,则有比较不同的情形。当革命还是反对帝国主义与大地主时,他们往往是能够保守中立或暂时的参加斗争的。尤其是从这个阶层出身而受过科学教育的知识分子,很多都能够这样做。

在抗日战争中,一部分大地主跟着一部分大资产阶级(投降派)已经投降日寇变为汉奸了,另一部分大地主跟着另一部分大资产阶级(顽固派),虽然还留在抗战营垒内,亦已非常的动摇。但是许多中小地主出身的开明绅士即带有若干资本主义色彩的地主们,尚有抗日的积极性,尚值得团结他们一道抗日。

\subsection{二、资产阶级}

资产阶级有带买办性的大资产阶级和民族资产阶级的区别。

带买办性的大资产阶级是直接为帝国主义的外国资本家服务并为他们所豢养的阶级,他们和农村中的半封建势力有着千丝万缕的联系。因此,在中国革命史上,大资产阶级历来不是中国革命的动力,而是中国革命的对象。

但因为中国带买办性的大资产阶级是分属于各个帝国主义的,在各个帝国主义间的矛盾尖锐地对立着的时候,在革命主要是反对某一个帝国主义的时候,属于别的帝国主义系统之下的买办阶级也有可能在一定程度上与一定时间内参加当前的反帝国主义战线。但一到他们的主子起来反对中国革命时,他们也就立即反对革命了。

在抗日战争中,亲日派大资产阶级(投降派),已经投降或准备投降了。欧美派大资产阶级(顽固派)虽然尚留在抗日营垒内,亦是非常动摇着,他们就是一面抗日与一面反共的两面派人物。我们对于大资产阶级投降派的政策是把他们当作敌人看待,坚决的打倒他们。而对于大资产阶级的顽固派,则是革命的两面政策,即一方面是联合他们,因为他们还在抗日,还应该利用他们一点残余的抗日积极性;又一方面是同他们作坚决斗争,因为他们执行破坏团结的反共反人民的高压政策,没有斗争就会危害抗战与团结。

民族资产阶级是两重性的阶级。

一方面,民族资产阶级受帝国主义的压迫,及受封建残余的束缚,所以,他们同帝国主义与封建残余有矛盾。从这一方面说来,他们是革命的动力之一,在中国革命史上,他们也曾经表现过这种反帝国主义与反官僚军阀政府的积极性。

但是又一方面,由于他们在经济上、政治上的软弱性,由于他们同帝国主义与封建残余并未完全断绝经济上的联系,所以,他们又没有彻底反帝反封建的勇气。这种情形,特别在民众革命力量强大起来的时候,表现得最明显。

民族资产阶级的这种两重性,决定了他们在一定时期中和一定程度上能够参加反帝国主义与反官僚军阀政府的革命,他们可以成为革命的一种动力。而在另一时期,就有跟在大资产阶级后面,作为反革命的助手的危险。

但是在中国的民族资产阶级主要都是中等资产阶级,他们虽然在一九二七年以后一九三一年(九一八事变)以前跟随着大地主大资产阶级反对过革命,但是这个阶层基本上还没有掌握过政权,而受当政的大地主大资产阶级的反动政策所限制。在抗日时期内,这个阶层不但与大资产阶级投降派有区别,而且与大资产阶级顽固派也有区别,至今仍是我们的较好的同盟者,因此对于这个阶层采取慎重的政策是完全必要的。

\subsection{三、各种类型的小资产阶级}

其中有知识分子、有城市贫民、有职员、有手工业者与自由职业者、有小商人。

所有这些小资产阶级,同农民阶级,都受帝国主义、封建残余与大资产阶级的压迫,日益走向破产没落的境地。

因此,小资产阶级是革命的动力之一,是无产阶级的可靠的同盟者。小资产阶级也只有在无产阶级领导之下,才能得到解放。

我们现在就来分析一下各种类型的没有把农民包括在内的小资产阶级。

第一是知识分子和青年学生。

数十年来,中国已造成了一个很大的知识分子群与青年学生群。在这一群人中间,除去一部分接近帝国主义与大资产阶级并为他们服务而反对民众的知识分子外,一般是受帝国主义封建残余与大资产阶级的压迫,使他们遭受着失业、失学的威胁的。因此,他们有很大的革命性。他们或多或少的有了现代的科学知识,富于政治感觉,他们在现阶段的中国革命中能够起着先锋的与桥梁的作用。辛亥革命前的留学生运动、一九一九年的五四运动、一九二五年的五三十运动、一九三五年的一二九运动,就是显明的例证。尤其是广大的比较贫苦的知识分子与半知识分子,能够和工农一道,参加和拥护革命。马克思列宁主义思想在中国的广大传播与接收,首先也是在知识分子与青年学生中。革命力量的组织与革命事业的建设,离开革命的知识分子的参加,是不能成功的。

但是知识分子在其未与民众的革命斗争打成一片,在其未下决心为民众利益服务并使其生活群众化之时,他们的思想往往是空虚的,他们的行动往往是动摇的。因此,中国的广大革命知识分子虽有先锋的与桥梁的作用,但不是所有这些知识分子都能参加革命到最后胜利的,其中一部分,到了革命的紧急关头时,就往往脱离革命队伍采取消极态度,其中少数人竟会变成革命的敌人。陈独秀、张国焘就是少数知识分子的代表。知识分子往往有一种主观的个人主义自大性,这种缺点,只有在长期群众斗争中才能洗刷干净。

第二是城市贫民。

这个阶层中,包括破产了的手工业者、小贩、离乡别井到城市寻找职业而不得的农民,以及大群依靠不定劳动维持生活的苦力。他们是一个很大的群众,他们的地位大体上和贫农的地位相当,是一种半无产阶级。他们的地位推动他们起来拥护革命,并使他们容易接受无产阶级的领导,所以他们是很好的革命力量,和贫农一样,是无产阶级的天然的同盟者。

第三是职员。

工商业机关中的职员,国家机关与文化机关中的广大月薪生活者,都属于这一类。他们是依靠出卖精神劳动或技艺而生活的人,是不剥削他人的。他们受失业威胁又非常之大,因此,也是重要的革命力量。这一类人是一个相当广大的群众。经济建设、国家建设与文化建设,是不能离开他们的。

第四是手工业者与自由职业者。

手工业者是独立生产者,是一个很大的群众,是现时中国经济建设的一个担负者。他们不但遭受外国商品竞争的打击,而且无力摆脱商业高利贷资本的罗网,所以他们能够站在革命的方面,他们也是重要的革命力量之一。他们当中的一部分是雇用少数工人的,另一部分则是不雇工人的。这后一部分人,是更加可靠的同盟者。

自由职业者例如医生等人虽是他们在思想意识上常常受资产阶级的影响,但他们是与手工业者属于同一范畴的,也是社会生活不可缺少的部门,也是受帝国主义与封建势力压迫的,所以也可以成为革命的力量。

第五是小商人。

他们一般是受帝国主义与大资产阶级的压迫,且是一个很大的群众。这一阶层的下层分子,是不剥削别人劳动,而遭受高利贷剥削的,所以他们在革命中是一支有用的力量。只有那些剥削他人劳动而又同帝国主义买办阶级或封建残余有联系的上层分子,才是对革命表现动摇态度的人们。

\subsection{四、农民阶级}

农民在全国总人口中占百分之八十,是现时中国国民经济的主要担负者。

农民一般都是小资产阶级,但他们的内部是在激烈分化的过程中。

第一是富农。约占农村人口百分之五左右(连地主一起共占农村人口百分之十左右),被称为农村的资产阶级。中国的富农大多带有半封建性,并与城市资产阶级联系着。但革命政府不应把富农看成与地主无分别的阶层,不应过早采取打击富农经济的政策,因为富农的生产在一定时期中是不可缺少的。

第二是中农。在中国农村人口中约占百分之二十左右。中农一般不剥削别人,在经济上能够自给自足(但在年成丰收时能有些许盈余,有时也利用一点雇佣劳动或放一点小债),而受帝国主义、地主阶级与大资产阶级的剥削,除一部分富裕中农外,多是土地不足并没有政治权利的。中农不但能够坚决参加反帝革命与土地革命,并且是能够参加社会主义革命的,因此全部中农都可以成为无产阶级的可靠的同盟者,中农是很好的革命动力之一。中农态度的向背是决定革命胜负的因素,尤其在土地革命之后,中农成了农村中的大多数的时候是如此。

第三是贫农。中国的贫农连同雇农在内,约占农村人口百分之七十。贫农是没有土地或土地不足的广大农村群众,是农村中的半无产阶级,是中国革命的最广大的动力,是无产阶级的天然的和最可靠的同盟者,是中国革命队伍的主力军。中农和贫农都只有在无产阶级的领导之下,才能得到解放;而无产阶级也只有向中农、贫农结成坚固的联盟,才能领导革命到胜利,否则是不可能的。农民这个名称所包括的内容,主要的也正是指的中农和贫农。

\subsection{五、无产阶级}

中国无产阶级中,现代产业工人约占二百五十万至三百万。城市手工业雇佣劳动者约占千二百万,此外还有广大的农村无产阶级。

中国无产阶级有他的许多特出的优点,使他在中国革命中能够成为领导的力量。

中国无产阶级有那些特出的优点呢?

第一、中国无产阶级身受三重压迫(帝国主义、资产阶级、封建势力),而这些压迫的严重性与残酷性,是世界各民族中少见的。因此。他们在革命斗争中,比任何别的阶级来得特别坚决和特别彻底。在殖民地半殖民地的中国又没有西欧那样的社会改良主义的经济基础(但须注意,中国民族改良主义有时容易在一部分工人中发生影响),所以除极少数的工贼之外,整个阶级都是最革命的。

第二、中国无产阶级,开始走上革命的舞台,就在本阶级的革命政党——中国共产党领导之下,成为中国社会里最有觉悟性的阶级。

第三、中国无产阶级同广大农民有一种天然的联系(由于刚从农业破产出身的成分占大多数),便利于他们同农民结成亲密的革命联盟。

因此,虽然中国无产阶级有其不可避免的弱点,例如人数较少(同农民比较)年龄较轻(同资本主义国家的无产阶级比较)、文化水准较低(同资产阶级比较);然而他们终究成为中国革命的最基本的动力,中国革命如果没有无产阶级的参加与领导,就必然不能胜利。远之如辛亥革命,因为当时还没有无产阶级的自觉的参加,因为那时还没有共产党,所以流产了。近之如一九二五——二七年的大革命,因为这时有了无产阶级的自觉的参加,因为这时有了共产党,所以在一个时期内取得了很大的胜利。但又因为资产阶级后来背叛了他们同无产阶级的联盟,背叛了共同的革命纲领,同时也由于那时中国无产阶级及其政党还没有丰富的革命经验,结果又遭受了失败。抗战以来,因为无产阶级和共产党参加了抗日民族统一战线的领导。所以团结了全民族,发动了与坚持了伟大的抗战。

中国无产阶级,在共产党领导之下,完全懂得:他们自己虽然是一个最有觉悟性和最有组织性的阶级,但如果单凭自己一个阶级的力量,是不能胜利的,而要胜利,就必须在各种不同情形下团结一切可能的革命阶级与阶层,组织革命的统一战线。在中国社会的各阶级中,农民是工人阶级的坚固的同盟军,城市小资产阶级也是可靠的同盟军,民族资产阶级则是在一定时期中与一定程度上的同盟军,这是现代中国革命的历史所已经证明了的根本规律之一。

中国的殖民地与半殖民地地位,造成了中国农村中与城市中广大的失业人群。在这个人群中,有许多人被迫到没有任何谋生的正当途径,不得不找寻所谓不名誉的或不正当的职业过活,这就是乞丐、盗贼、流氓、娼妓与许多迷信职业家的来源。这个阶层是动摇的阶层,其中一部分容易被反动势力所收买;另一部份则颇有革命性。但是他们缺乏建设性,破坏有余而建设不足,就又成为流寇主义与无政府思想的来源。因此,应该善于引导他们,注意组织他们的革命性,而防止他们那种不正当的破坏性。

以上这些,就是我们对于中国革命动力的分析。

\subsection{第五节 中国革命的性质}

我们已经明白了中国社会的性质,亦即中国的特殊国情,这是解决中国一切革命问题的最基本的根据。我们又明白了中国革命的对象、中国革命的任务、中国革命的动力,这些都是由于中国社会的特殊性质,由于中国的特殊国情而发生的关于现阶段中国革命的基本问题。在明白了所有这些之后,那么,我们就可以明白现阶段中国革命的另一个基本问题,即中国革命的性质是什么了。

现阶段的中国革命究竟是一种什么性质的革命呢?资产阶级民主主义的革命,还是无产阶级社会主义的革命呢?显然的,不是后者,而是前者。

既然中国社会还是一个殖民地、半殖民地、半封建的社会,既然中国革命的敌人主要的还是帝国主义与半封建势力,既然中国革命的任务是在推翻这两个主要敌人的民族革命与民主革命;而推翻这两个敌人的革命动力,有时还有民族资产阶级及一部分大资产阶级的参加,即使大资产阶级背叛革命而成了革命的敌人,革命的锋芒也不是向着一切资本主义与资本主义的私有财产,而是向着帝国主义与封建独占。即然如此,所以现阶段中国革命的性质,不是无产阶级社会主义的,而是资产阶级民主主义的。

但是现时中国的资产阶级民主主义革命,已不是旧式的一般的资产阶级民主主义革命,这种革命已经过时了,而是新式的特殊的资产阶级民主主义革命。这种革命正在中国与一切殖民地半殖民地国家发展起来,我们称这种革命为新民主主义的革命。这种新民主主义的革命是世界无产阶级社会主义革命的一部分,他是坚决反对帝国主义即国际资本主义的。他在政治上是几个革命阶级联合起来对于帝国主义者及汉奸反动派的革命民主专政,反对把中国社会造成资产阶级专政的社会。他在经济上是把帝国主义者及汉奸反动派的大资本大企业收归国家经营,把大土地分配给农民所有,同时扶助私人的中小企业,并不废除富农经济。因此,这种新式的民主革命,虽然一方面是替资本主义扫清道路,但在另一方面又是替社会主义创造前提。中国现时的革命阶段,是为了终结殖民地、半殖民地、半封建社会与建立社会主义社会之间的一个过渡阶段,是一个新民主主义的新的革命过程。这个过程是从第一次世界大战与俄国十月革命之后才发生的,在中国则是从一九一九年五四运动开始的。所谓新民主主义的革命,就是在无产阶级领导之下的人民大众反帝反封建的革命,就是各革命阶级统一战线的革命。中国必须经过这个革命,才能进一步发展到社会主义革命,否则是不可能的。

这种新民主主义的革命,与欧美各国历史上的民主革命大不相同,他不造成资产阶级专政,而是造成各革命阶级统一战线的专政。在抗日战争中,应该建立的抗日民主政权,乃是抗日民族统一战线的政权,他既不是资产阶级的“一党专政”,也不是无产阶级的“一党专政”,而是抗日民族统一战线的“几党专攻”,只要是赞成抗日又赞成民主的人们,不问属于何党何派,都有参加政权的资格。

这种新民主主义的革命也与社会主义革命不相同,他只推翻帝国主义与汉奸反动派,而不推翻任何尚能参加反帝反封建的一切资本主义成分。

这种新民主主义革命,同孙中山在一九二四年所宣布的三民主义革命(国民党第一次全国代表大会宣言)在基本上是一致的。因为孙中山在这个宣言上说:“近世各国所谓民权制度,往往为资产阶级所专有,适成为压迫平民之工具,盖国民党之民权主义,则为一般平民所共有,非少数人所得而私也。”又说:“凡本国人及外国人之企业,或有独占的性质,或规模过大为私人之力所不能办者,如银行、铁路、航路之属,由国家经营管理之,使私有资本制度不能操纵国民之生计,此则节制资本之要旨也。”孙中山又在其遗嘱上提出“必须唤起民众与以平等待我之民族共同奋斗”的内政外交的根本原则。所有这些,乃是区别于适应于旧的国际国内环境之旧民主主义的三民主义,而改造成了适应于新的国际国内环境之新民主主义的三民主义。中国共产党在一九三七年九月二十二日发表宣言,声明“三民主义为中国今日之必需,本党愿为其彻底实现而奋斗”,就是指的这种三民主义,而不是任何别的三民主义。这种三民主义即是孙中山三大政策,即联俄、联共与农工政策的三民主义,在新的国际国内条件下,离开三大政策的三民主义,就不是革命的三民主义(关于共产主义与三民主义只是在民主革命政纲上基本相同,而其它一切方面则均不相同,这一问题这里不来说他)。

这样,就使中国的资产阶级民主革命,无论就其斗争阵线(统一战线)来说,就其国家组成来说,均不能忽视无产阶级、农民阶级、知识分子与其他小资产阶级的地位,谁要是想撇开中国的无产阶级、农民阶级、知识分子与其他小资产阶级,就一定不能解决中华民族的命运,一定不能解决中国的任何问题。中国现阶段革命所要造成的民主共和国,一定要是一个工人、农民与知识分子在其内面占一定地位起一定作用的民主共和国,换言之,即是一个工人、农民、知识分子、小资产阶级与其他一切反帝反封建分子之革命联盟的民主共和国。这种共和国的彻底完成,只有在无产阶级的政策领导之下才有可能。

\subsection{第六节 中国革命的前途}

在将现阶段中国社会的性质,中国革命的对象、任务、动力与性质这些基本问题弄清楚了之后,那么,对于中国革命的前途问题,就是说,中国资产阶级民主革命与无产阶级社会革命的关系问题,中国革命的现在阶段与将来阶段的关系问题,也就容易明白了。

因为既然现阶段中国资产阶级民主主义的革命,不是一般的旧式的资产阶级民主主义革命,而是特殊的新式的民主主义革命,而是新民主主义的革命,而中国革命现时又是处在二十世纪四十与五十年代的新的国际环境中,即处在社会主义向上高涨,资本主义向下低落的国际环境中,处在第二次帝国主义大战中与第二次世界革命的前夜,那么,中国革命的前途,不是资本主义的,而是社会主义的,也就没有疑义了。

没有问题,现阶段的中国革命既然是为了变更现在的殖民地、半殖民地、半封建社会的地位,即为了完成一个新民主主义的革命而奋斗,那么,在革命胜利之后,因为革命肃清了资本主义发展道路上的障碍物,资本主义经济在中国社会中会有一个相当程度的发展,是可以想象到的,也是不足为怪的。资本主义有个相当程度的发展,这是经济落后的中国在民主革命胜利之后不可避免的结果,当然,不容否认这只是中国革命的一方面结果,不会是它的全部结果。中国革命的全部结果是:一方面有资本主义因素的发展,又一方面有社会主义因素的发展。这种社会主义因素是什么呢?就是无产阶级与共产党在全国政治势力中的比重的增长,农民、知识分子与小资产阶级或者已经或者可能承认无产阶级与共产党的领导权。所有这一切,便都是社会主义的因素,加以国际环境的有利,便使中国资产阶级民主革命的最后结果,避免资本主义前途,实现社会主义前途,不能不具有极大可能性了。

\subsection{第七节 中国革命的两重任务与中国共产党}

总结本章各节所述,我们可以明白,整个中国革命是包含着两重任务的,这就是说,中国革命是包括资产阶级民主主义性质的革命(新民主主义的革命)与无产阶级社会主义性质的革命,现在阶段的革命与将来阶段的革命这样两重任务的。而这两重革命任务的领导,都是担负在中国无产阶级的政党——共产党的双肩之上,离开了中国共产党的领导,任何革命都不能成功。

完成中国资产阶级民主主义革命(新民主主义革命),并准备在一切必要条件具备之时把他转变到社会主义的革命阶段上去,这就是中国共产党光荣的伟大的全部革命任务,每个共产党员都应为此而奋斗,绝对不能半途而废。有些幼稚的共产党员,以为我们只有现在阶段的民主主义革命的任务,没有将来阶段的社会主义革命的任务,或者说,现在的革命或土地革命即是社会主义的革命,应该着重指出,这些观点都是错误的。每个共产党员须知,整个中国的共产主义运动,是包括民主革命与社会革命两个阶段在内的全部革命运动,这是两个性质不同的革命过程,只有完成前一个革命过程才可能去完成后一个革命过程。民主革命是社会革命的必要准备,社会革命是民主革命的必然趋势。而一切共产主义者的最后目的,则是在于力争社会主义与共产主义社会的最后的完成。只有认清民主革命与社会革命的区别,同时又认清二者的联系,才能正确的领导中国革命。

领导中国民主主义革命与中国社会主义革命这样两个伟大的革命到彻底的完成,除了中国共产党之外,是没有任何一个别的政党(不论资产阶级或小资产阶级政党)能够担负的。而中国共产党则从自己建党的一天起,就把这样的两重任务放在自己的双肩之上了,并且已为此而艰苦奋斗了整整十八年。

这样的任务是非常光荣的,但同时也就是非常艰苦的,没有一个全国范围的、广大群众性的,思想上、政治上、组织上完全巩固的,布尔塞维克的中国共产党,是不能完成的。因此,如何建设这样一个共产党,乃是每一个共产党员的责任。

以下,我们就来逐步讨论中国共产党的建设问题。
