
\title{延安民众讨汪拥蒋大会通电}
\date{一九四〇年二月一日}
\thanks{这篇通电在列入毛选的时候,题目改为《向国民党的十点要求》。毛选所载的文章,基本内容没有删改,只把个别句子修改,例如删除了“拥护蒋委员长”,或者“我国民政府”修改为“国民政府”,务求不再露出当时任何承认国民党政府法统的含意。但是有趣的是这个通电的上款及下款所列的一大串名字或者团体、报章的名称,都全部删除。}
\maketitle



\mxname{重庆国民政府林主席,蒋委员长,各院,部,会,中国国民党中央党部,本事委员会各部,国民参政会,反侵略大同盟,中苏文化协会,战地党政委员会,生活书店,商务印书馆,中华书局,青年记者学会,文艺界抗敌协会,中央社,国新社,大公报,新华日报,扫荡报,中央日报,三民主义青年团,各省省政府,省参议会,省党部,抗敌后援会,报界协会,各大日报,四川、云南、汉中各大学,西安、柱林两行营,各战区司令长官,各集团军总司令,全国抗战将士,抗战同胞均鉴:}

二月一日延安讨汪拥蒋大会,义愤激昂,一致决议,讨伐汪精卫卖国投降,拥护蒋委员长抗战到底。为挽救时局危机争取抗战胜利起见,谨陈救国大计十端,愿我国民政府,各党各派,抗战将士,全国同胞采纳而实行之。

一曰全国讨汪。查汪逆收集党徒,通敌叛国,订立卖国密约,为虎作伥,固国人皆曰可杀。然此乃公开之汪精卫尚未语于暗藏之汪精卫也。若夫暗藏之汪精卫,则招摇过市,窃据要津,匿影藏形,深入社会,贪官污吏,实为其党徒,磨擦专家,皆属其部下,若无全国讨汪运动,从都市以至乡村,从上级以至下级,动员党、政、军、民、报、学各界,悉起讨汪,则汪党不绝,汪祸长留,外引敌人,内施破坏,其为害有不堪设想者。宜由政府下令,唤起全国人民讨汪,有不奉行者,罪其官吏,务绝汪党,投畀豺虎。此应请采纳实行者一。

二曰加紧团结。今之论者不言团结而言统一,其意盖谓惟有取消共产党,取消八路军新四军,取消陕甘宁边区,取消各地方抗日力量,始谓之统一。不知共产党八路军新四军陕甘宁边区,乃全国主张统一之最力者。主张西按事变和平解决者,非共产党八路军新四军与边区乎?发起抗日民族统一战线主张建立统一民主共和国而身体力行之者,非共产党八路军新四军与边区乎?立于国防之最前线抗御敌军十七个师团,屏障中原西北,保卫华北、江南,坚决实行三民主义与抗战建国纲领者,非共产党八路军新四军与边区乎?盖自汪精卫倡言反共亲日以来,张君劢叶青等妖人和之以笔墨,反共派、顽固派和之以磨擦,假统一之名,行独霸之实。弃团结之义,造磨擦之端。司马昭之心,固已路人皆知矣。若夫共产党八路军新四军与边区,则坚决提倡真统一,反对假统一,提倡合理的统一,反对不合理的统一,提倡实际上的统一,反对形式上的统一。非统一于投降而统一于抗战,非统一于分裂而统一于团结,非统一于倒退而统一于进步,以抗战、团结、进步三者为基础之统一乃真统一,乃合理统一,乃实际统一。舍此而求统一,无论出何花样,弄何玄虚,均属南辕北辙,实属未敢茍同。至于各地方抗日力量,则宜一体爱护,不宜厚此薄彼,信任之,接济之,扶掖之,奖励之,待人以诚而去其诈,待人以宽而去其隘,诚能如此,则茍非别有用心之徒,未有不团结一致而纳于统一国家之轨道者。统一必以团结为基础,团结必以进步为基础。惟进步乃能团结,惟团结乃能统一,实为不易之定论。此应请采纳实行者二。

三曰励行宪政。训政多年,毫无结果,物极必反,宪政为先。然而言论不自由,党禁未开放,一切犹是反宪政之行为。以此制宪,何殊官样文章。以此行宪,何异一党专制。当此国难深重之秋,若犹不思变计,则日、汪肆扰于外,奸徒破坏于内,国脉民命,岌岌可危矣。我政府宜即开放党禁,扶植舆论,以为诚意推行宪政之表示,昭大信于国民,启新国之气运,诚未有急于此者。此应请采纳实行者三。

四曰制止磨擦。自去年二月倡导所谓“限制异党活动办法”以来,限共溶共反共之声遍于全国,惨案迭起,血花乱飞。犹以为未足,去年十月复有所谓“异党问题处理办法”。其在西北、华北、华中区域,复有所谓“对于处理异党问题实施方案”。论者谓己由政治限共进入军事限共之期,言之有据,何莫不然。盖所谓限共者,反共也。反共者,日、汪之诡计,亡华之毒策也。于是群情惊疑,奔走相告,以为又将重演十年前之惨剧。演变所极,湖南则有平江惨案,河南则有确山惨案,河北则有张荫梧进攻八路军,山东则有秦启荣消灭游击队,鄂东有程汝怀惨杀共产党员五六百之众,陇东有中央军大举进攻八路驻防军之举,而最近山西境内复演出旧军攻击新军并联带侵犯八路阵地之惨剧。此等现象,不速制止,势将同归于尽,抗战胜利云乎哉?宜由政府下令处罚一切制造惨案份子,并昭示全国不许再有同类事件发生,以利团结抗战。此应请采纳实行者四。

五曰保护青年。近在西安附近设立集中营,将西北、中原各省之进步青年七百余人拘系一处,施以精神与肉体之奴役,形同囚犯,惨不忍闻,青年何辜,遭此荼毒?夫青年乃国家之精华,进步青年尤属抗战之至宝。信仰为人人之自由,而思想乃绝非武力所能压制者,过去十年文化围剿之罪恶,彰明较著,奈何今日又欲重蹈之乎?政府宜速申令全国,保护青年,取消西安附近之集中营,严禁各地侮辱青年之暴举。此应请采纳实行者五。

六曰援助前线。最前线之抗日有功军队,例如八路军新四军及其它军队,待遇最为菲薄,衣单食薄,弹药不继,医疗不备。而奸人反肆无忌惮,任意污蔑,无数不负责任毫无常识之谰言,震耳欲聋。有功不赏,有劳不录,而构陷愈急,毒谋愈肆。此皆将士寒心,敌人拊掌之怪现象,断乎不能容许者也。宜由政府一面充分接济前线有功军队,一面严禁奸徒污蔑构陷,以励军心而利作战。此应请采纳实行者六。

七曰取缔特务机关。特务机关之横行。时人比诸唐之周兴,来俊臣,明之魏忠贤,刘瑾。彼辈不注意敌人而以对内为能事,杀人如麻,贪贿无艺,实谣言之大本营,奸邪之制造所。使通国之人重足而立,侧目而视者,无过于此辈穷凶极恶之特务人员。为保存政府威信起见,极宜实行取缔,加以改组,确定特务机关之任务为专对敌人及汉奸,以回人心,而培国本。此应请采纳实行者七。

八曰取缔贪官污吏。抗战以来,有发国难财至一万万元之多者,有讨小老婆至八、九个之多者。举凡兵役也,公债也,经济之统制也,灾民、难民之救济也,无不为贪官污吏借以发财之机会。国家有此一群虎狼,无怪乎国事不可收拾。人民怨愤已达极点,而无人敢暴露其凶残。为挽救国家崩溃之危机起见,极宜断行有效办法,彻底取缔一切贪官污吏。此应请采纳实行者八。

九曰实行总理遗嘱。总理遗嘱有云:“余致力国民革命凡四十年,其目的在求中国之自由平等,积四十年之经验,深知欲达到此目的,必须唤起民众。”大哉言乎,我四万万五千万人民实闻之矣。顾诵读遗嘱者多,遵循遗嘱者少,背弃遗嘱者奖,实行遗嘱者罚,事之可怪,宁有逾此?宜由政府下令,有敢违背遗嘱,不务唤起民众而反践踏民众者,处以背叛孙总理之罪。此应请采纳实行者九。

十曰实行三民主义。政府与蒋委员长已三令五申,责成全国实行三民主义,

甚盛事也,顾言者谆谆,听者藐藐,无数以反共为第一任务之人,放弃抗战工作,人民起而抗日,则多方压迫制止,此放弃民族主义者也。官吏不给予人民以丝毫民主权利。此故弃民权主义者也。视人民之痛苦若无睹,此放弃民生主义者也。在此辈人员眼中,三民主义不过口头禅,而有真正实行之者,不笑之日多事,即治之以严刑,由此怪象丛生,信仰扫地。亟宜再颁明令,严督全国实行,有违令者。从重治罪,有遵令者,优于奖励,则三民主义庶乎有实行之日,而抗日事业即建立胜利之基。此应请采纳实行者十。

凡此十端,皆救国之大计,抗日之要图。当此敌人谋我愈急,汪逆极端猖獗之时。心所谓危,不敢不告。倘蒙采纳施行,抗战幸甚,中华民族解放事业幸甚。追切陈词,愿闻明教。

延安民众讨汪拥蒋大会主席团毛泽东,王明,张闻天,林伯渠,吴玉章,王稼穑,康生,陈云,邓发,李富春,高岗,萧劲光,张浩,张邦英,许光达,孟庆树,谭政,唐洪晨,高朗亭,冯文彬,管瑞才暨全体民众三万人同叩东。
