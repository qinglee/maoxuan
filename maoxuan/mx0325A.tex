
\title{论联合政府}
\date{一九四五年四月二四日在中国共产党第七次全国代表大会上的政治报告}
\thanks{这篇文章在收入毛选的时候作了非常大幅度的修改。在第二节《国际形势与国内形势》,有很多段落是大力称许英国和美国在反法西斯战争中的作用,几乎全部删去。在第四节《中国共产党的政策》,有一小段主张建立中华民主共和国联邦,还有“在这个联邦基础上组织中央政府”的话,亦全部删去。在同一节,删去一小段,这段强调新民主主义并不是要实行社会主义:不仅如此,整节在很多地方都在事后补加上“无产阶级领导”的话。窜改的目的是为了使原文中那种强调发展资本主义,强调社会主义是另一个历史时期的立场,变成为处处强调在“无产阶级领导下”发展新民主主义经济,使人觉得毛泽东会迅速从新民主主义转变为社会主义。此外删去的还有很多的,例如有一大段保证共产党即使在国民大会中得到多数,也不会组织一党政府;在发展工业的那部分,把欢迎外资的话删去了。}
\maketitle


\section{一 中国人民的基本要求}

同志们!盼望很久的我们党的第七次全国代表大会,现在开会了,我代表中央委员会向你们作报告。目前的时局,要求我们的大会讨论与决定许多重大问题。然后,我们将向中国人民说明我们的意见。如果他们同意我们的意见,我们就协同他们动手去做。

我们的大会是在这种情况之下开会的:中国人民在其对于日本侵略者作了将近八年的坚决的英勇的不屈不挠的奋斗,经历了无数的艰难困苦与自我牺牲之后,出现了这样的新局面:整个世界上反对法西斯侵略者的神圣的正义的战争,已经取得了有决定意义的胜利,中国人民配合同盟国打败日本侵略者的时机,己经迫近了。但是中国现在仍然不团结,日本侵略者仍然在压迫我们,中国仍然存在着严重的危机。在此种情况下,我们应该怎样做呢?毫无疑义,中国急需团结各党各派及无党无派的代表人物在一起,成立民主的临时的联合政府,以便实行民主的改革,克服目前的危机,动员与统一全中国的抗日力量,有力地和同盟国配合作战,打败日本侵略者,使中国人民从日本侵略者手中解放出来。然后,在广泛的民主基础之上,召开国民代表大会,成立包括更广大范围的各党各派与无党无派代表人物在内的同样是联合性质的民主的正式政府,领导解放后的全国人民,将中国建设成为一个独立、自由、民主、统一与富强的新国家。一句话,走团结与民主的路线,打败侵略者,建设新中国。

我们认为只有这样做,才是反映了中国人民的基本要求。因此,我的报告,主要地就是讨论这些要求。中国应否成立民主的联合政府,已成了中国人民及同盟国舆论界十分关心的问题,因此,我的报告,将着重地说明联合政府问题。

中国共产党在八年抗日战争中的工作,己经克服了很多的困难,获得了巨大的成绩。但是在目前形势下,在我党与人民面前,尚存在着严重的困难。目前的时局,要求我党进一步从事紧急的与更加切实的工作,继续地克服困难,为完成中国人民的基本要求而奋斗。

\section{二 国际形势与国内形势}

中国人民能不能实现我们在上面提出的那些基本要求呢?我们认为两种可能性都存在,依靠中国人民觉悟、团结与努力的程度来决定。但是目前的国际国内形势,都对中国人民提供了有利的条件。中国人民如能很好地利用这些条件,积极地坚决地再接再厉地向前奋斗,战胜侵略者与建设新中国,是毫无疑义的。中国人民应当加倍努力,为完成自己的神圣任务而奋斗。

目前的国际形势是怎样的呢?

和中国及外国一切反动派的预料相反,英美苏三大民主国一直是团结的。她们之间,过去存在过,将来还可能发生某些争议,但是团结终究是统治一切的。这是一个决定一切的条件,克里米亚会议最后地证明了这一点。这个条件是在世界历史的重大转变关节——二十世纪四十年代产生的。在法西斯侵略战争爆发成为威胁全世界人民的战争的时候,实际上帮助法西斯侵略者反对英美苏团结的反动势力,从许多主要国家(不是一切国家)的政治舞台上大批地被推落下去,赞成英美苏团结的反法西斯势力占了上风,这个条件就产生了。自从世界上出现了这个条件,世界的面目就改观了。整个法西斯势力及其在各国的游魂,必须被消灭。国际间的重大问题,必须以三大国或五大国为首的协议来解决。各国内部问题,无例外地必须按照民主原则来解决。世界将引向进步,决不是引向反动。这些就是我们这个世界的新面目。当然应该提起充分的警觉,了解到历史的若干暂时的甚至是严重的曲折,可能还会发生;许多国家中不愿看见本国人民与外国人民获得团结、进步与解放,不愿看见英美苏中法继续团结领导世界新秩序的世界分裂主义者的反动势力,还是强大的;谁要是忽视了这些,谁就将在政治上犯错误。但是,历史的总趋向已经确定,不能改变了,世界的新面目已经出现了。

这个新面目,仅仅不利于法西斯和在实际上帮助法西斯的各国反动派(中国的也在内)。对于一切国家的人民及其有组织的民主势力,则都是福音。

人民,只有人民,才是创造世界的动力。苏联人民创造了强大力量,充当了打倒法西斯的主力军。英、美、中、法四大国及其它反法西斯同盟国的人民的伟大努力,使打倒法西斯成为可能。法西斯被打倒以后,各国人民将建设一个巩固的与持久的和平世界。四月二十五日在旧金山举行的联合国会议,将是这种和平的起点。

战争教育了人民,人民将赢得战争,赢得和平,又赢得进步,这就是目前世界新形势的规律。

这一新形势,与第一次世界大战及在其后的所谓“和平”时代,是大不相同的。在那时,还没有现在这样的苏联,也没有现在这样的英、美、中、法及其它反法西斯同盟各国的人民的觉悟程度,自然也就不能有三大国或五大国为首的现在这样的世界团结。我们现在是处在完全新的局面之下。现在有的是:觉悟了与团结了并且正在更加觉悟与更加团结的世界人民以及人民的有组织的力量,这就规定了世界历史车轮所要走向的目标和到此目标所要选取的道路。

法西斯侵略国家被打败,总的和平局面出现了以后,并不是说就没有了斗争。广泛地散布着的法西斯残余势力,一定还要捣乱。反法西斯侵略战争的阵营中,存在着反民主势力,他们仍然要压迫人民。所以,国际和平实现以后,反法西斯的人民大众与法西斯残余势力之争,民主与反民主之争,仍将充满世界的大部分地方。只有经过长期的努力,克服了法西斯残余势力及反民主势力,才能有最广泛的人民的胜利。到达这一天,决不是很快与很容易的,但是必然要到达这一天。反法西斯战争——正义的第二次世界大战的胜利,给这个战后人民斗争的胜利开辟了道路。也只有这后一种斗争胜利了,巩固的与持久的和平才得了保障。这就是世界人民的光明前途。

由于英美苏三大民主国及其它欧洲国家的团结,最后地打败法西斯德国的战争很快就可结束,红军已攻击柏林,这个希特勒的神经中枢,不日可下。英美法盟军正在最后地打击希特勒残军。意大利人民发动了起义。这一切,将最后地打败希特勒。希特勒被打败以后,将在世界上出现这样的局面:解放欧洲,并立即增强着解放亚洲的可能性,从而使亚洲获得解放。

英美中三大国团结在反对日本侵略者的事业上。由于中国人民在八年战争中的长期艰苦的奋斗,英国在东方的协同作战,特别是美国在太平洋上的胜利,使战争迫近了日本的大门。日本侵略者已处于极端的不利地位,它的军心民心已发生了更大的动摇。但是它还有力量,它正在准备持久的挣扎,并希望通过中国及同盟国内部的动摇分子谋取妥协的和平。但是一切太平洋国家全体人民的利益,均要求完全消灭日本侵略者。苏联已经废除了苏日中立条约,这件事给了中国人民及太平洋各国人民以极大的兴奋。在这种种情况下,我们应该这样说:到达最后地完全地消灭日本侵略者,还有一段艰难的路程,我们决不可轻敌;但是胜利的把握是更大了,我们一定能胜利。

中国人民从来也没有遇到过象现在这样有利的国际条件,这个条件鼓励中国人民支持了长期的战争。

目前中国的国内形势是怎样的呢?

中国的长期战争,使中国人民付出了并且还将再付出重大的牺牲。但是同时,正是这个战争,锻炼了并且还将再锻炼英勇斗争的中国人民。这个战争促进了中国人民的觉悟与团结到了这样的程度:不但在中国古代没有过,就是近百年来中国人民的一切伟大斗争,也没有一次比得上的。在中国人民面前,不仅存在着强大的民族敌人,而且存在着强大的实际上帮助民族敌人的国内反动势力,这是一方面。但是另一方面,中国人民不但已经有了较之过去任何时候要高的觉悟程度,而且有了强大的中国解放区与日益高涨着的全国性的民主运动,所有这些,就是国内的有利条件。如果说,中国近百年来一切人民斗争都遭到了失败或挫折,而这是因为缺乏国际国内的必要的条件,那么,这一次,就不同了,比较以往历次,一切必要的条件更具备了,避免失败与取得胜利的可能性充分地存在。如果我们能够团结全国人民,努力奋斗,并给以适当的指导,我们就会有胜利。

中国人民团结起来打败侵略者与建设新中国的信心,现在是极大地增强了,中国人民克服一切困难,实现其具有伟大历史意义的基本要求的时机,已经到来了。这一点还有疑义吗?我以为没有疑义了。

这些,就是目前国际与国内的一般形势。

\section{三 抗日战争中的两条路线}

\subsection{中国问题的关键}

谈到国内形势,我们还应对中国抗日战争加以具体的分析。

中国是全世界反法西斯战争中五个最大国家之一,是在亚洲大陆上反对日本侵略者的主要国家。中国人民不但在抗日战争中起了与还将起极大的作用,而且在保障战后世界和平上将起极大的作用,在保障东方和平上则将起决定的作用。中国人民在其八年抗日战争中,为了自己的解放,为了帮助各同盟国的胜利,曾经作了伟大的努力。这种努力,主要地是属于中国人民方面的。中国军队的广大官兵,在前线流血战斗;中国的工人、农民、知识界、产业界,在后方努力工作;海外华侨,输财助战;一切抗日政党,除了那些反人民分子外,均对战争有所尽力。总之,中国人民以自己的血与汗和日本侵略者英勇地奋战了八年之久。但是多年以来,中国反动分子造作谣言,蒙蔽舆论,不使中国人民在抗日战争中所起作用的真相暴露于世。同时,对于中国八年抗日战争的各项经验,也还没有人作出全面的总结来。因此,我们的大会,应当对此作出适当的总结,借以教育人民并为我党决定政策的根据。

提到总结经验,那么,大家可以很清楚地看到,中国存在着两条不同的指导路线,一条是能够打败日本侵略者的,一条是不但不能打败日本侵略者,而且在某些方面说来,它是在实际上帮助日本侵略者危害抗日战争的。

由于国民党政府所采取的对日作战的消极政策与对内积极摧残人民的反动政策,招致了战争的失败,国土的大部沦陷,财政经济的危机,人民的被压迫,人民生活的痛苦,民族团结的被破坏,妨碍了动员与统一一切中国人民的抗日力量进行有效的战争,妨碍了中国人民的觉醒与团结。但是,中国人民的觉醒与团结的运动并没有停止,它是在日本侵略者与国民党政府的双重压迫之下曲折地发展着。两条路线:国民党政府压迫中国人民实行消极抗战的路线与中国人民觉醒与团结起来实行人民战争的路线,很久以来,就明显地在中国存在着,这就是一切中国问题的关键所在。

\subsection{走着曲折道路的历史}

为了使大家明了何以这个两条路线问题是一切中国问题的关键所在,必须回溯一下我们抗日战争的历史。

中国人民的抗日战争,是在曲折的道路上发展起来的。这个战争,还是在一九三一年就开始了。一九三一年九月十八日,日本侵略者占领沈阳,几个月内,就把东三省占领了,国民党政府采取了不抵抗政策。但是东三省的人民,东三省的一部分爱国军队,在中国共产党领导或协助之下,违反国民党政府的意志,组织了东三省的抗日义勇军,从事英勇的游击战争。这个英勇的游击战争,曾经发展到很大的规模,中经许多困难挫折,但是始终没有被敌人消灭。一九三二年,日本侵略者进攻上海,国民党内的一派爱国分子,又一次违反国民党政府的意志,率领十九路军,抵抗了日本侵略者的进攻。一九三三年,日本侵略者进攻热河、察哈尔,国民党内的又一派爱国分子,第三次违反国民党政府的意志,组织了抗日同盟军,从事抵抗。但是一切这些抗日战争,除了中国人民、中国共产党、其他民主派别及海外华侨予以援助之外,国民党政府根据其不抵抗政策,是不给任何援助的。相反地,上海,察哈尔两次抗日行动,均被国民党政府一手破坏了。一九三三年,十九路军在福建成立了人民政府,也被国民党政府破坏了。

那时的国民党政府为什么采取不抵抗政策呢?主要的原因,在于国民党政府在一九二七年破坏了国共两党的合作与中国人民的团结。

一九二四年,孙中山先生接受了中国共产党的建议,召集了有共产党人参加的第一次国民党全国代表大会,订出了联俄、联共、扶助农工的三大政策,建立了黄埔军校,实现了国共两党及各界人民的民族统一战线,因而在一九二五年,扫荡了广东的反动势力,在一九二六至一九二七年,举行了胜利的北伐战争,占领了长江流域及黄河流域,打败了北洋军阀政府,发动了广大的在中国历史上是空前的人民解放斗争。但是到了一九二七年春夏之交,正当北伐战争向前发展的紧要关头,这个代表中国人民解放事业的国共两党及各界人民的民族统一战线及其一切革命政策,就被国民党当局的叛卖性的反人民的“清党”政策与屠杀政策所破坏了。昨天的同盟者——中国共产党与中国人民,被看成了仇敌,昨天的敌人——帝国主义者与封建主义者,被看成了同盟者。就是这样,背信弃义地向着中国共产党与中国人民来一个突然的袭击,生气蓬勃的中国大革命就被葬送了。从此以后,内战代替了团结,独裁代替了民主,黑暗的中国代替了光明的中国。

但是,中国共产党与中国人民并没有被吓倒、被征服、被杀绝,他们从地下爬起来,揩干净身上的血迹,掩埋好同伴的尸首,他们又继续战斗了。他们高举起革命的大旗,举行了武装的抵抗,在中国广大的区域内,组织了人民的政府,实行了土地的改革,创造了人民的军队——中国红军,保存了与发展了中国人民的革命力量。被国民党反动分子所抛弃的孙中山先生的革命的三民主义,由中国人民、中国共产党及其他民主分子继承下来了。

到了日本侵略者打入东三省以后,中国共产党就在一九三三年,向一切进攻苏区与红军的国民党军队提议:在一、停止进攻;二、给予人民以自由权利;三、武装人民这样三个条件之下,订立停战协定,以便一致抗日。但是国民党当局拒绝了这个提议。

从此以后,一方面,是国民党政府的内战政策越发猖狂。另一方面,是中国人民要求停止内战一致抗日的呼声越发高涨,各种人民爱国组织,在上海及其它许多地方建立起来。一九三五至一九三六年,长江南北各地的红军主力,在我党中央领导之下,经历了千辛万苦,移到了西北,并和西北红军汇合在一起。就在这两年,中国共产党适应新的情况,决定并执行了抗日民族统一战线的新的完整的政治路线,以团结抗日与建立新民主主义共和国为奋斗目标。一九三五年十二月九日,北平学生群众,在我党领导之下,发动了英勇的爱国运动,成立了中华民族解放先锋队,并使此种爱国运动推广到了全国各大城市。一九三六年十二月十二日,国民党内部主张抗日的两派爱国分子——东北军与十七路军,联合起来,勇敢地反对国民党当局的对日妥协与对内屠杀的反动政策,举行了有名的西安事变。同时,国民党内的其他爱国分子,也不满意国民党当局的当时政策。在此种形势下,国民党当局被迫放弃了内战政策,承认了人民的要求。以西安事变和平解决为时局转换的枢纽,形成了在新形势下的国内团结,发动了全国的抗日战争。在芦沟桥事变的前夜,即一九三七年五月,我党召集了一个具有历史意义的全国代表会议。在这个会议上,批准了我党自一九三五年以来的新的政治路线。

从一九三七年七月七日芦沟桥事变到一九三八年十月武汉失守这一时期内,国民党政府的对日作战是比较努力的。在这个时期内,由于日本侵略者的大举进攻与全国人民民族义愤的高涨,使得国民党政府将其政策的重点放在反对日本侵略者身上,形成了全国军民抗日战争的高潮,一时出现了生气蓬勃的新气象。当时全国人民,我们共产党人,其他民主党派,都对国民党政府寄与极大的希望,就是说,希望它乘此民族艰危、人心振奋的时机,厉行民主改革,将孙中山先生的革命三民主义付诸实施。可是这个希望是失败了。就在这两年,一方面,有比较积极的抗战;另一方面,政府当局仍旧反对发动广大民众参加的人民战争,仍旧限制人民自动团结起来进行抗日与民主的活动。一方面,政府对待中国共产党及其他抗日党派的态度比较过去有了一个改变;另一方面,仍旧不给各党派以平等地位,并多方限制他们的活动,许多爱国政治犯并没有释放。主要的是国民党政府仍旧保持其自一九二七年发动内战以来的寡头专制形态,未能建立举国一致的民主的联合政府。

这一时期内,我们共产党人就指出中国抗日战争的两条路线:或者是人民的全面的战争,这样就会胜利;或者是压迫人民的片面的战争,这样就会失败、我们又指出:战争将是长期的,必然要遇到许多艰难困苦;但是由于中国人民的努力,最后胜利必归于中国人民。

\subsection{人民战争}

这一时期内,中国共产党领导的移到了西北的中国红军主力,改编为中国国民革命军第八路军,留在长江南北各地的中国红军游击部队,则改编为中国国民革命军新编第四军,相继开赴华北华中作战。内战时期的中国红军,保存了并发展了北伐时期黄埔军校及国民革命军的民主传统,曾经扩大到几十万人。由于国民党政府在南方各根据地内的残酷的摧毁、万里长征的消耗及其它原因,数量减少到几万人。于是有些人看不起这枝军队,以为抗日主要依靠国民党。但是人民是最好的鉴定人,他们知道八路军新四军此时数量虽小,质量很高,只有它才能执行真正的人民战争,它一旦开到抗日的前线,和那里的广大人民相结合,其前途是无限的。人民是正确的,当我在这里做报告的时候,我们的正式军队已发展到了九十一万人,民兵发展到了二百二十万以上。不管现在我们的正式军队比起国民党现存的军队来(包括中央系与地方系)在数量上还要少几十万,但是按其所抗击的敌伪军的数量与其所担负的战场的广大说来,按其战斗力说来,按其有广大的人民、民兵与自卫军配合作战说来,按其政治质量与其内部统一团结步调一致说来,它已经成了中国抗日战争的主力军。

这个军队之所以有力量,是因为参加这个军队的一切人们,具有自觉的纪律,他们不是为着少数人的或狭隘集团的私利,而是为着正义的人民战争,为着广大人民群众的利益,为着全民族的利益,而结合,而战斗的。紧紧地和中国人民站在一起,全心全意地为中国人民服务,这就是这个军队的唯一宗旨。

在这个宗旨下面,这个军队具有一往无前的精神,它要压倒一切敌人,而决不被敌人所屈服。不论在任何艰难困苦的场合,只要还存在一个人,这个人就要继续战斗下去。

在这个宗旨下面,这个军队有一个很好的内部团结与外部团结。在内部:官兵之间,上下级之间,军事工作、政治工作与后勤工作之间,在外部:军民之间,军政之间,我友之间,均必须是团结的;一切妨害这些团结的现象,均在必须克服之列。

在这个宗旨下面,这个军队有一个正确的争取敌军官兵与处理俘虏的政策。凡属投诚的,反正的,或在放下武器后愿意参加反对共同敌人的敌伪军人,一概表示欢迎,并给予适当的教育。一切俘虏,不许杀害、虐待与侮辱。

在这个宗旨下面,这个军队形成了为人民战争所必需的一系列的战略战术,它善于按照变化着的具体条件从事机动灵活的游击战争,也善于作运动战。

在这个宗旨下面,这个军队形成了为人民战争所必需的一系列的政治工作,为团结我军,团结友军,团结人民,瓦解敌军与保证战斗胜利而斗争。

在这个宗旨下面,在游击战争的条件下,全军都可以并且已经是这样做了:利用战斗与训练的间隙,从事粮食与日用必需品的生产,达到军队自给,半自给,或部分自给之目的。借以克服经济困难,改善军队生活及减轻人民负担。在各个军事根据地上,也利用了一切可能性建立了许多小规模的军事工业。

这个军队之所以有力量,还由于有人民自卫军与民兵这样广大的群众武装组织,和它一道配合作战。在中国解放区,一切青年,壮年,甚至老年的男人与女人,都在自愿的民主的与不脱离生产的原则下,组织在抗日人民自卫军之中。自卫军中的精干分子,除加入军队及游击队者外,则组织在民兵之中。没有这些群众武装力量的配合,要战胜日本侵略者是不可能的。

这个军队之所以有力量,还由于它将自己划分为主力兵团与地方兵团两部分,前者可以随时执行超地方的作战任务,后者则固定在协同民兵、自卫军保卫地方与进攻当地敌人的任务上,这种划分,取得了人民的真心拥护。如果没有这种正确的划分,例如说,如果只注意主力兵团的作用,忽视地方兵团的作用,那么,在中国解放区的条件下,要战胜日本侵略者也是不可能的。在地方兵团方面,组织了许多经过良好训练,在军事、政治、民运各项工作上说来都是比较更健全的武装工作队,深人敌后之敌后,打击敌人,发动民众的抗日斗争,借以配合各个解放区正面战线的作战,收到了很大的成效。

在中国解放区,在民主政府领导之下,号召一切抗日人民组织在工人的、农民的、青年的、妇女的、文化的及其他职业与工作的团体之中,热烈地从事援助军队的各项工作,例如动员人民参加军队,替军队运输粮食,优待抗日军人家属,帮助军队解决物质困难。在这方面更重要的,是动员游击队、民兵与自卫军,展开袭击运动与爆炸运动,侦察敌情,清除奸细,运输伤兵与保护伤兵,直接帮助了军队的作战。同时,全解放区人民又热烈地从事政治、经济、文化、卫生各项建设工作。在这方面最重要的,是动员全体人民从事粮食与日用品的生产,并使一切机关学校,除少数特殊情形者外,一律于工作或学习之暇,从事生产自给,以配合人民与军队的生产自给,造成了伟大的生产热潮,借以支持长期的抗日战争,尤为中国解放区的特色。在中国解放区,敌人的摧残是异常严重的。水、旱、虫灾,亦时常发生。但是,解放区民主政府领导全体人民,有组织地克服了与正在克服着各种困难,灭蝗、治水、救灾的伟大群众运动,收到了史无前例的成绩,使抗日战争能够长期地坚持下去。总之,一切为着前线,一切为着打倒日本侵略者与解放中国人民,这就是中国解放区全体军民的总口号,总方针。

这就是真正的人民战争。只有这种人民战争,才能战胜民族敌人。国民党之所以失败,就是因为它拼命地反对人民战争。

中国解放区的军队一旦得到新式武器的装备,它就会更加无敌,能够最后地打败日本侵略者了。

\subsection{两个战场}

中国战场,一开始就分为两个战场:国民党战场与解放区战场。

一九三八年十月武汉失守后,日本侵略者停止了向国民党战场的战略性的进攻,逐渐将其主要军事力量移到了解放区战场;同时,针对着国民党政府的失败情绪,声言愿意和它谋取妥协的和平,并将卖国贼汪精卫诱出重庆,在南京成立伪政府,实施民族的欺骗政策。从此时起,国民党政府开始了政策上的变化,将其重点由对外逐渐转移到对内。这首先表现在它的军事政策上,采取了对日作战的消极政策,保存军事实力,而把作战的重担放在解放区战场上,让日本人大举进攻解放区,自己则坐山观虎斗。

一九三九年,国民党政府采取了反动的所谓“限制异党活动办法”,将抗战初期给予人民及抗日党派的某些权利,一概收回。从此时起,在国民党统治区内国民党政府将一切民主党派,首先与主要的是将中国共产党,打人地下。在国民党统治区各个省份的监狱与集中营内,充满了共产党人、爱国青年及其他民主战士。从一九三九年起的五年之内,直至一九四三年秋季为止,国民党政府举行了三次大规模的“反共高潮”,分裂国内的团结,造成严重的内战危险。有名的“解散”新四军及歼灭皖南新四军部队九千人的事变,就是发生在这个时期内。直到现时为止,国民党军队向解放区军队进攻的事件还未停止,并且看不出任何准备停止的征象。在此种情况下,一切诬蔑与谩骂,都从国民党反动分子的嘴里喷了出来。什么“奸党”、“奸军”、“奸区”,什么“破坏抗战、危害国家”,都是这些反动分子为着反对中国人民的得意的制造品。一九三九年七月七日,中国共产党中央委员会发表宣言,针对着当时的危机,提出了这样的口号:“坚持抗战,反对投降;坚持团结,反对分裂;坚持进步,反对倒退。”在这些适合时宜的口号之下,在五年之内,有力地打退了三次反动的反人民的“反共高潮”,克服了当时的危机。

在这几年内,国民党战场实际上没有严重的战争。日本侵略者刀锋,主要地向着解放区战场。到一九四三年,侵华日军的百分之六十四及伪军的百分之九十五,为解放区战场担负着。国民党战场所担负的,不过日军的百分三十六及伪军的百分之五而已。

一九四四年,日本侵略者举行打通大陆交通线的作战了,国民党战场表现出手足无措,毫无抵抗能力,几个月内,就将河南、湖南、广西、广东等省广大区域沦于敌手。仅在此时,两个战场分担抗敌的比例,才起了一些变化。然而就在我做这个报告的时候,在侵华日军(满洲的尚不在内)四十个师团,五十八万人中,解放区战场抗击的是二十二个半师团,三十二万人,占了百分之五十六;国民党战场抗击的,不过十七个半师团,二十六万人,仅占百分之四十四。伪军的分担则完全无变化。

还应指出,数达八十万以上的伪军(包括伪正规军与伪地方武装在内),大部分是国民党将领率部投敌,或由国民党投敌军官所组成的。国民党反动分子事先即供给这些伪军以所谓“曲线救国”的叛国谬论,事后又在精神上与组织上支持他们,使他们配合日本侵略者反对中国解放区。此外,则动员大批军队封锁与进攻陕甘宁边区及敌后各解放区,其数量达到了七十九万七千人之多。这种严重情形,在国民党政府的新闻封锁政策下,很多的中国人外国人都无法知道。很多人只知道南斯拉夫有一个米海洛维奇,而不知道中国有几十个米海洛维奇。

\subsection{中国解放区}

中国解放区,现在领有九千五百五十万人口。其地域,北起内蒙,南至海南岛,大部分敌人所到之处,都有八路军、新四军及其它人民军队的活动。在这个广大的中国解放区内,包括了十九个大的解放区,其地域则包括了辽宁、热河、察哈尔、绥远、陕西、甘肃、宁夏、山西、河北、河南、山东、江苏、浙江、安徽、江西、湖北、湖南、广东、福建等省,在这些省份中有些是大部分,有些是小部分,而以延安为各个解放区的指导中心。在这个广大的解放区内,黄河以西的陕甘宁边区,其人口一百五十万,不过是十九个解放区中的一个,而且除了浙东、琼崖两区之外,按其人口说来,它是一个最小的。有些人不明此种情形,以为所谓中国解放区,主要就是陕甘宁边区,这是一个误会,是由于国民党政府的封锁政策造成的。在这个广大的解放区内,实行了抗日民族统一战线的全部必要的政策,建立了或正在建立民选的共产党人和各党各派及无党无派代表人物合作的政府,亦即地方性的联合政府,全体人民的力量都动员起来了。所有这一切,使得中国解放区在强敌压迫之下,在国民党军队的封锁与进攻之下,在毫无外援之下,能够屹然树立,一天一天发展,缩小敌占区,扩大解放区,成为民主中国的模型,成为配合同盟国驱逐日本侵略者解放中国人民的重心。中国解放区的军队——八路军、新四军及其它人民军队,不但在对日战争的作战上,起了英勇的模范的作用,在执行抗日民族统一战线的各项民主政策上,也是起了模范作用的。

一九三七年九月二十二日,中国共产党中央委员会发表宣言:承认孙中山先生的三民主义为中国今日之必需,本党愿为其彻底实现而奋斗。这一宣言,在中国解放区是完全实践了。

\subsection{国民党统治区}

国民党内的主要统治集团,坚持独裁统治,实行了消极的抗日政策与反人民的国内政策。这样,就使得它的军队缩小了一半以上,并且大部分几乎丧失了战斗力;使得它自己和广大人民之间造成了深刻的裂痕,造成了民生凋敝,民怨沸腾,民变蜂起的严重危机;使得它在抗日战争中的作用,不但是极大地减少了,并且变成了动员与统一中国人民一切抗日力量的障碍物。

为什么在国民党主要统治集团领导下会产生这种严重情况呢?因为这个集团所代表的利益是中国的大地主、大银行家、大买办阶层的利益。这个极端少数的反动阶层,垄断着国民党政府管辖之下的军事、政治、经济、文化的一切重要的机构。他们将保全自己少数人的利益放在第一位,而把抗日放在第二位。他们也说“民族至上”,但是他们的行为却不符合于民族中大多数人民的要求。他们也说“国家至上”,但是他们所指的国家,就是大地主、大银行家、大买办阶层的封建法西斯独裁国家,并不是人民大众的民主国家。因此,他们惧怕人民起来,惧怕民主运动,惧怕认真的动员全民的抗日战争。这就是他们之所以采取对日作战的消极政策,对内的反人民、反民主、反共的反动政策之总根源。要问他们为什么采取这样的两面政策,例如,一面虽在抗日,一面又采取消极的作战政策,并且还被日本人经常选择为诱降的对象:一面在口头上宣称要发展中国经济,一面又在实际上扶助官僚资本,亦即大地主、大银行家、大买办的资本,垄断着中国的主要经济命脉,而残酷地压迫农民,压迫工人,压迫小资产阶级与自由资产阶级;一面在口头上宣称实行“民主”,“还政于民”,一面又在实际上残酷地压迫人民的民主运动,不愿实行丝毫的民主改革;一面在口头上宣称:“共党问题为一政治问题,应用政治方法解决”,一面又在军事上、政治上、经济上残酷地压迫中国共产党,把共产党看成他们的所谓“第一个敌人”,而把日本侵略者看成“第二个敌人”,并且每天都在积极地准备内战,处心积虑地要消灭共产党;一面在国民党内部,他们口头上宣称应有“精诚团结”,一面又在实际上鼓励中央系军队欺压地方系军队(所谓“杂牌军”),鼓励专制派欺压民主派,鼓励各派之间互相对立,以利其独裁统治;一面在口头上宣称要建立一个“近代国家”,一面又在实际上拼死命保持那个大地主、大银行家、大买办的封建法西斯独裁国家;一面和苏联在形式上保持外交关系,一面又在实际上采取仇视苏联的态度;一面依赖英美的援助,一面又反对英美的自由主义;一面同美国孤立派合唱“先亚后欧论”,藉以延长法西斯德国也就是延长一切法西斯的寿命,延长自己对于中国人民的法西斯统治的寿命,一面又在国际大家庭里投机取巧,把自己打扮成为反法西斯的英雄;要问如此种种的自相矛盾的两面政策从何而来,就是来自大地主、大银行家、大买办社会阶层这个总根源。

但是国民党是一个复杂的政党。它是被这个代表大地主、大银行家、大买办阶层的反动集团所统治,所领导,但是整个国民党并不等于这个反动集团。它有许多领袖人物不属于这个集团,而且被这个集团所打击、排斥或轻视。它有更多的干部与党员群众及三青团团员群众并不满意这个集团的领导,而且有些甚至是反对它的领导的。这种情形,在被这个反动集团所统制的国民党的军队,国民党的政府机关,国民党的经济机关与国民党的文化机关中,都是存在着。在这些军队及机关里,包藏着广大的进步的民主分子。这个反动集团,其中又分为几派,互相斗争,并不是一个严密的统一体。把国民党看成清一色的反动派,无疑是很不适当的。

\subsection{比较}

中国人民从中国解放区与国民党统治区,获得了明显的比较。

难道还不明显吗?两条路线,人民战争的路线与反对人民战争的消极抗日的路线,其结果:一条是胜利的,即使处在中国解放区这种环境恶劣与毫无外援的地位。另一条是失败的,即使处在国民党统治区这种极端有利与取得外国接济的地位。

国民党政府把自己的失败归咎于缺乏武器。但是试问:缺乏武器的是国民党的军队呢?还是解放区的军队?中国解放区的军队是中国军队中武器最缺乏的军队,他们只能从敌人手里夺取及在最恶劣条件下自己制造其武器。

国民党中央系军队的武器,不是比起地方系军队来要好得多吗?但是比起战斗力来,中央系却多数劣于地方系。

国民党拥有广大的人力资源,但是在它的错误的兵役政策下,人力补充却极端困难。中国解放区处在被敌人分割与战斗频繁的情况之下,却因适合人民需要的民兵与自卫军制度之普遍实施,又防止了对于人力资源的滥用与浪费,使人力动员源源不竭。

国民党拥有粮食丰富的广大地区,人民每年供给七千万至一万万市担的粮食,但是大部分被经手人员中饱了,致使国民党的军队经常缺乏粮食,士兵饿得面黄肌瘦。中国解放区的主要部分隔在敌后,遭受敌人烧杀抢三光政策的摧残,其中有些是象陕北这样贫瘠的区域,但是却能用自己动手、发展农业生产的方法,很好地解决了粮食问题。

国民党区域经济危机极端严重,工业大部分破产了,连布匹这样的日用品也要从美国运来。中国解放区却能用发展工业的方法,自己解决布匹及其它日用品。

国民党区域工人,农民、店员、公务人员、知识分子及文化人,生活痛苦,达于极点。中国解放区的全体人民都有饭吃,有衣穿,有事做,有书读,有些地方做到了丰衣足食。

利用抗战发国难财,官吏即商人,贪污成风,廉耻扫地,这是国民党区域的特色之一。艰苦奋斗,以身作则,工作之外,还要生产,奖励廉洁,禁绝贪污,这是中国解放区的特色之一。

国民党区域剥夺人民的一切自由。中国解放区则给予人民以充分的自由。

怪别人,还是怪自己?怪外国缺乏援助,还是怪国民党政府的独裁统治与腐败无能?这难道还不明白吗?

\subsection{“破坏抗战、危害国家”的是谁?}

真凭实据地破坏了中国人民的抗战与危害了中国人民的国家的,难道不正是国民党政府吗?这个政府一心一意地打了整十年的内战,将刀锋向着同胞,置一切国防事业于不顾,又用不抵抗政策送掉了东四省。日本人打进本部来了,仓皇应战,从芦沟桥退到了贵州省。但是国民党人却说:“共产党破坏抗战,危害国家”(见一九四三年九月国民党十一中全会的决议案)。唯一的证据,就是共产党联合了各党各派各界人民创造了英勇抗日的中国解放区。这些国民党人的逻辑,与中国人民的逻辑是这样的不相同,无怪乎很多问题都讲不通了。

两个问题:

第一个:究竟什么原因使得国民党政府抛弃了从黑龙江到芦沟桥,又从芦沟桥到贵州省这样广大的国土与这样众多的人民?难道不是由于国民党政府所采取的不抵抗政策、消极的抗日政策与反人民战争的国内政策吗?

第二个,究竟什么原因使得中国解放区战胜了敌伪军长期的残酷的进攻,从民族敌人手里恢复了这样广大的国土,解放了这样众多的人民?难道不是由于人民战争的正确路线吗?

\subsection{所谓“不服从政令、军令”}

国民党政府还经常以“不服从政令、军令”责备中国共产党,但是我们只能这样说:幸喜中国共产党人还保存了中国人民的普通常识,没有服从那些实际上是把中国人民艰难困苦地从日本人手里夺回来的中国解放区再送交日本人的所谓“政令、军令”,例如一九三九年的所谓“限制异党活动办法”,一九四一年的所谓“解散新四军”与“退至旧黄河以北”,一九四三年的所谓“解散中国共产党”,一九四四年的所谓“限期取消十个师以外的全部军队”,以及在最近谈判中提出来的所谓将军队及地方政府移交给国民党,其交换条件是不许成立联合政府,只许收容几个共产党员到国民党独裁政府里去做官,并将此种办法称之为国民党政府的“让步”等等。幸喜我们没有服从这些东西,替中国人民保存了一片干净土,保存了一枝英勇抗日的军队,难道中国人民不应该庆贺这一个“不服从”吗?难道国民党政府自己用自己的法西斯主义的政令与失败主义的军令,将黑龙江至贵州省的广大土地、人民送交日本人,还觉得不够吗?除了日本人及反动派欢迎这些“政令、军令”之外,难道还有什么爱国的有良心的中国人欢迎这些东西吗?没有一个不是形式的而是实际的、不是独裁的而是民主的联合政府,能够设想中国人民会允许中国共产党人,擅自将这个获得了解放的中国解放区与抗日有功的人民军队,交给失败主义与法西斯主义的国民党独裁政府吗?假如没有中国解放区及其军队,中国人民的抗日事业还有今日吗?我们民族的前途还能设想吗?

\subsection{内战危险}

迄今为止,国民党内的主要统治集团,坚持着独裁与内战的反动方针。有很多迹象表明,他们早已准备,龙其现在正在准备这样的行动:等候某一个同盟国的军队在中国大陆上驱除日本侵略者到了某一程度时,他们就要发动内战。他们并且希望某些同盟国将领们在中国境内执行斯科比将军在希腊所执行的职务,他们对于斯科比及希腊反动政府的屠杀事业,表示欢呼。他们企图把中国抛回到一九二七至一九三六年的国内战争的大海里去。国民党主要统治集团现在正在所谓“召开国民大会”及“政治解决”的烟幕之下,偷偷摸摸地进行其内战的准备工作。如果国人不加注意,不去揭露它的阴谋,阻止它的准备,那么,会有一个早上,要听到内战的炮声的。

\subsection{谈判}

为着打败日本侵略者与建设新中国,为着防止内战,中国共产党在取得了其它民主派别的同意之后,于一九四四年九月间开会的国民参政会上,提出了立即废止国民党专政,成立民主的联合政府一项要求。无疑地,这项要求是适合时宜的,几个月内,获得了广大人民的响应。

关于如何废止一党专政、成立联合政府以及实行必要的民主改革等项问题,我们和国民党政府之间曾经有了多次谈判,但是我们的一切建议,都遭到了国民党政府的拒绝。不但一党专政不愿废止,联合政府不愿成立,任何迫切需要的民主改革,例如取消特务机关,取消镇压人民自由的反动法令,释放政治犯,承认各党派合法地位,承认解放区,撤退封锁与进攻解放区的军队等,一项也不愿实行。就是这样,使得中国的政治关系处在非常严重的局面之下。

\subsection{两个前途}

从整个形势看来,从上述一切国际国内实际情况的分析看来,我请大家注意,不要以为我们的事业,一切都将是顺利的,美妙的。不,不是这样,事实是好坏两个可能性,好坏两个前途都存在着。继续独裁统治,不许民主改革;不是将重点放在反对日本侵略者上面,而是放在反对人民上面;即使日本侵略者被打败,中国仍可能发动内战,将中国拖回到痛苦重重的不独立,不自由,不民主,不统一,不富强的老状态里去,这是一个可能性,这是一个前途。这个可能性,这个前途,依然存在,并不因为国际形势之好,国内人民觉悟程度之增长及有组织的人民力量之发展,它就似乎没有了,或自然地消失了。希望中国实现这个可能性,实现这个前途的,在中国是国民党内的反人民集团,在外国是那些怀抱帝国主义思想的反动分子。这是一方面,这是必须注意与不可不注意的一方面。

但是,另一方面,同样是从整个形势看来,从上述一切内外情况的分析看来,使我们更有信心地与更有勇气地去把握第二个可能性,第二个前途。这就是克服一切困难,团结全国人民,废止独裁统治,实行民主改革,巩固与扩大抗日力量,彻底打败日本侵略者,将中国建设成为一个独立、自由、民主、统一与富强的新国家。希望中国实现这个可能性,实现这个前途的,在中国是广大人民,中国共产党及其他民主分子与民主派别,在外国是一切以平等地位待我之民族,外国的进步分子,外国的人民大众。

我们清楚地懂得,在我们和中国人民面前,还有很大的困难,还有很多的障碍物,还要走很多的迂回路程。但是我们同样地懂得,任何困难与障碍物,我们和中国人民一定能够克服,而使中国的历史任务获得完成。竭尽全力地去反对第一个可能性,争取第二个可能性,反对第一个前途,争取第二个前途,是我们和中国人民的伟大任务。国际国内的主要条件,主要方面,是有利于我们与中国人民的。所有这些,我在前面已经说得很清楚了。我们希望国民党当局,鉴于世界大势之所趋,中国人心之所向,毅然改变其错误的现行政策,使抗日战争获得胜利,使中国人民少受痛苦,使新中国早日诞生。须知不论怎样迂回曲折,中国人民独立解放的任务总是要完成的,而且这种时机已经到来了。一百多年来无数先烈所怀抱的宏大志愿,一定要由我们这一代人去实现,谁要阻止,到底是阻止不了的。

\section{四 中国共产党的政策}

上面,我已将中国抗日战争中的两条路线,给了一个分析。这样的一个分析,在我看来是完全必要的。因为在广大的中国人中间,至今还有很多人不明白中国抗日战争中的具体情况,在国民党统治区及在国外,由于国民党政府的封锁政策,很多人被蒙住了眼睛。在一九四四年中外新闻记者考察团及美军观察组来到中国解放区以前,那里的许多人对于解放区几乎是什么也不知道的。一九四五年一月二十八日“纽约时报”说“解决中国共产党问题的最好办法,莫如任令人们往来于两个区域之间,许多误会就会消失。”可是国民党政府非常害怕,一九四四年的一次新闻记者团回去之后,立即将大门堵上,不许一个新闻记者再来解放区。对于国民党区域的真相,国民党政府也是同样的加以封锁。因此,我感到我们有责任将“两个区域”的真相尽可能使人们弄清楚。只有在弄清中国的全部情况之后,才有可能了解中国的两个最大政党——中国共产党与中国国民党的政策何以有这样的不同,何以有这样的两条路线之争。只有这样,才会使人们了解,两党的争论,不是如有些人们所说不过是一些不必要的,不重要的,或甚至是意气用事的争论,而是关系着几万万人民生死问题的原则的争论。

在目前中国时局的严重形势下,中国人民,中国一切民主党派及民主份子,一切关心中国时局的外国人民以及许多同盟国的政府,都希望中国的分裂局面重趋于团结,都希望中国能实行民主改革,都愿意知道中国共产党对于解决当前许多重大问题上所持的政策。我们的党员对于这些,当然更加关心。

我们的抗日民族统一战线的政策历来是明确的,八年战争中考验了这些政策。我们的大会应该对此作出结论,作为今后奋斗的指针。

下面,我就来说明我党在为解决中国问题而得出的关于重要政策方面的若干确定的结论。

\subsection{我们的一般纲领}

为着动员与统一中国人民一切抗日力量,彻底消灭日本侵略者,并建立独立自由、民主、统一与富强的新中国,中国人民、中国共产党及一切抗日民主党派迫切地需要一个互相同意的共同纲领。

这种共同纲领,可分为一般性的与具体性的两部分,我们先来说一般性的纲领,然后再说到具体性的纲领。

在彻底消灭日本侵略者与建设新中国的大前提之下,在中国现阶段上,我们共产党人在这样一个基本点上是和中国人口中最广大的成分相一致的,就是说:第一,中国既不应该是一个由大地主大资产阶级专政的、封建的、法西斯的、反人民的国家制度,因为这种反人民的制度,已由国民党主要统治集团在其十八年的统治中表现出完全破产了。第二,中国也不可能、因此就不应该企图建立一个纯粹自由资产阶级旧式民主主义专政的国家。因为在中国,一方面,自由资产阶级在经济上与政治上至今还表现得很软弱;另一方面,中国早已产生了一个觉悟了的,在中国政治舞台上表现了强大能力的,领导了广大的农民阶级、小资产阶级、知识分子及其他民主分子的中国无产阶级及其领袖——中国共产党这样的新条件。第三,在中国的现阶段上,在中国人民的任务还是反对民族压迫与封建压迫,在中国社会经济的必要条件还不具备时,中国人民也不可能、因此就不应该企图实现社会主义的国家制度。

那么,我们的主张是什么呢?我们主张在彻底消灭日本侵略者之后,建立一个以全国绝对大多数人民为基础的统一战线的民主联盟的国家制度,我们把这样的国家制度称之为新民主主义的国家制度。

这是一个真正适合中国人口中最广大成分的要求的国家制度。因为第一,它取得了与可能取得数百万产业工人,数千万手工工人与雇佣农民的同意;其次,也取得了与可能取得占中国人口百分之八十,即在四万万五千万人口中占了三万万六千万的农民阶级的同意;又其次,也取得了与可能取得广大的小资产阶级,自由资产阶级,开明绅士,及其他爱国分子的同意。

自然,这些阶级之间也有其不同的要求,在这一点上,它们之间仍然是存在着矛盾的,例如劳资之间的矛盾。抹杀这种不同要求,抹杀这种矛盾,是虚伪的与错误的。但是,这种不同要求,这种矛盾,在整个新民主主义制度的阶段上,不会也不应该使之发展到超过共同要求之上。这种不同要求与这种矛盾,可以获得调节。在这种调节下,可以共同完成新民主主义国家的政治、经济与文化的各项建设。

我们主张的新民主主义的政治,就是推翻外来的民族压迫,废止国内的封建主义的与法西斯主义的压迫,并且主张不在推翻与废止这些之后又建立一个旧民主主义的政治制度,而是建立一个联合一切民主阶级的统一战线的政治制度,我们的这种主张,是和孙中山先生的主张完全一致的。孙先生在其所著国民党第一次代表大会的宣言里说:“近世各国所谓民权制度,往往为资产阶级所专有,适成为压迫平民之工具。盖国民党之民权主义,则为一般平民所共有,非少数人所得而私也。”这是孙先生的伟大政治指示,中国人民,中国共产党及其他一切民主分子,必须服从这个指示而坚决实行之,并和一切违背与敌对这一指示的任何人们与任何集团作坚决的斗争,借以保护与发扬这个完全正确的新民主主义的政治原则。

新民主主义的政权构成,应该采取民主集中制,由各级人民代表大会决定大政方针,选举政府。它是民主的,又是集中的,就是说,在民主基础上的集中,在集中指导下的民主。只有这个制度,既能表现广泛的民主,使各级人民代表大会有最高的权力,又能集中处理国事,使各级政府能集中地处理被各级人民代表大会所委托的一切事务,并保障人民的一切必要的民主活动。

在新民主主义的国家问题与政权问题上,包含着联邦的问题。中国境内各民族,应根据自愿与民主的原则,组织中华民主共和国联邦,并在这个联邦基础上组织联邦的中央政府。

军队及其他武装力量,是新民主主义的国家权力机关的重要部分,没有它们,就不能保卫国家。新民主主义的一切武装力量,如同其他权力机关一样,是属于人民与保护人民的,它们和一切属于少数人与压迫人民的旧式军队、旧式警察等等完全不同。

我们主张的新民主主义的经济,也是符合于孙先生的原则的。在土地问题上,孙先生主张“耕者有其田”。在工商业问题上,孙先生在上述宣言里这样说:“凡本国人及外国人之企业,或有独占性质,或规模过大为私人之力所不能办者,如银行、铁路、航路之属,由国家经营管理之,使私有资本制度不能操纵国民之生计,此即节制资本之要旨也。”在现阶段上,对于经济问题,我们完全同意孙先生的这些主张。

有些人们怀疑中国共产党人不赞成发展个性,不赞成发展私人资本主义,不赞成保护私有财产,其实都是过虑。民族压迫与封建压迫残酷地束缚着中国人民的个性发展,束缚着私人资本主义的发展与破坏着广大人民的财产。我们主张的新民主主义制度的任务,则正是解除这些束缚与停止这种破坏,保障广大人民能够自由发展其在共同生活中的个性,能够自由发展那些不是“操纵国民生计”,而是有益于国民生计的私人资本主义经济,保障一切正当的私有财产。

按照孙先生的原则与中国革命的经验,在现阶段上,中国的经济,必须是由国家经营,私人经营与合作社经营三者组成的。而这个国家经营的所谓国家,一定不应该是“少数人所得而私”的国家,而一定要是“为一般平民所共有”的新民主主义的国家。

新民主主义的文化,同样应该是“为一般平民所共有”的,即是说,民族的、科学的、大众的文化,决不应该是“少数人所得而私”的文化。

上述一切,就是我们共产党人在现阶段上,在整个资产阶级民主革命的阶段上所主张的一般纲领,或基本纲领。对于我们的社会主义与共产主义制度的将来纲领,或最高纲领说来,这是我们的最低纲领。实行这个纲领,可以把中国从现在的国家性质与社会性质上向前推进一步,即是说,从殖民地半殖民地与半封建的国家性质与社会性质,推进到新式资产阶级民主主义的国家性质与社会性质,在彻底消灭日本侵略者以后,可以建立一个具有这种性质的独立、自由、民主、统一与富强的国家。

实行这个纲领,还没有把中国推进到社会主义。这不是一个由于什么人在主观上想做或不想做这种推进的问题,而是一个由于在客观上中国的政治条件与社会条件不许可人们这样做的问题。

我们共产党人从来不隐瞒自己的政治主张,我们的将来纲领,或最高纲领,是要将中国推进到社会主义与共产主义的,这是确定的与毫无疑义的。我们的党的名称和我们的马克思主义的宇宙观,明确地指明了这个将来的,无限光明的,无限美妙的与最高理想的方向。我们每个人入党之时,心目中就悬着为现在的新式资产阶级民主主义革命而奋斗与为将来的无产阶级社会主义革命而奋斗这样两个明确的目标,而不顾那些共产主义敌人们的无知的与卑劣的敌视、诬蔑、谩骂或讥笑,对于这些,我们必须给以坚决的排击,对于那些善意的怀疑者,则不是排击而是给以善意的与耐心的解释,所有这些,都是异常清楚、异常确定与毫不含糊的。

但是,一切中国共产党人,一切中国共产主义的同情者,必须为着现时阶段的目标而奋斗,为着反对民族压迫与封建压迫,为着使中国人民脱离殖民地、半殖民地、半封建的悲惨命运与建立新式资产阶级民主主义性质的,新民主主义性质的,亦即孙中山先生革命三民主义性质的独立、自由、民主、统一与富强的中国而奋斗。我们果然是这样做了,我们共产党人,协同广大的中国人民,为此而英勇奋斗了二十四年。

对于任何一个共产党人及其同情者,如果不为这个目标而奋斗,如果看不起这个资产阶级民主革命而对它稍许放松,稍许怠工,稍许表现其不忠诚,不热情,不准备付出自己的鲜血与生命,而空谈什么社会主义与共产主义,那就是无意地,或有意地,或多或少地背叛社会主义与共产主义,就不是一个自觉的与忠诚的共产主义者。只有经过民主主义,才能到达社会主义,这是马克思主义的天经地义。而在中国,为民主主义而奋斗的时间还是长期的,没有一个新民主主义的联合统一的国家,没有新民主主义的国家经济的发展,没有广大的私人资本主义经济与合作社经济的发展,没有民族的科学的大众的文化即新民主主义文化的发展,没有几万万人民的个性解放与个性发展,一句话,没有一个新式资产阶级性质的彻底的民主革命,要想在殖民地半殖民地半封建的废墟上建立起社会主义来,那只是完全的空想。

有些人不了解共产党人为什么不但不怕资本主义,反而提倡它的发展。我们的回答是这样简单:拿发展资本主义去代替外国帝国主义与本国封建主义的压迫,不但是一个进步,而且是一个不可避免的过程,它不但有利于资产阶级,同时也有利于无产阶级。现在的中国是多了一个外国的帝国主义与一个本国的封建主义,而不是多了一个本国的资本主义,相反地,我们的资本主义是太少了。说也奇怪,有些中国资产阶级代言人不敢正面提出发展资本主义的主张,而要转湾抹角地来说这个问题。另外有些人,则甚至一口否认中国应该让资本主义有一个广大的发展,而说什么一下发展到社会主义,什么要将三民主义与社会主义“毕其功于一役”。很明显地,这类现象,有些是反映着中国自由资产阶级的软弱性,有些则是反映大地主大资产阶级对于民众的欺骗手段。我们共产党人根据自己对于马克思主义的社会发展规律的认识,明确地知道,在中国的条件下、在新民主主义的国家统治下,除了国家自己的经济与劳动人民的个体经济及合作社经济之外,定要让私人资本主义经济获得广大发展的便利,才能有益于国家与人民,有益于社会的向前发展。对于中国共产党人,任何的空谈与欺骗,是不会让它迷惑我们的清醒头脑的。

有些人怀疑共产党人“承认三民主义为中国今日之必需,本党愿为其彻底实现而奋斗”,似乎不是忠诚的。这是由于不了解我们所承认的孙中山先生在一九二四年国民党第一次代表大会宣言里所解释的三民主义的基本原则,和我党在现阶段的纲领即最低纲领里的若干基本原则,是互相一致的。应当指出,孙先生的这种三民主义,和我党在现阶段上的纲领,只是在若干基本原则上是一致的东西,并不是完全一致的东西。我党的新民主主义的纲领,比之孙先生的,当然要完备得多,特别是孙先生死后这二十年中中国革命的发展,使我党新民主主义的理论、纲领及其实践,有了一个极大的发展,今后还将有更大的发展。但是,孙先生的这种三民主义,按其基本性质说来,是一个新民主主义的三民主义,是一个和在此以前的旧三民主义相区别的新三民主义,当然这是“中国今日之必需”,当然“本党愿为其彻底实现而奋斗”。对于中国共产党人,为本党的最低纲领而奋斗与为孙先生的革命三民主义即新三民主义而奋斗,在其基本上(不是在一切上)是一件事情,并不是两件事情。因此,不但在过去,在现在,已经证明,而且在将来还要证明:中国共产党人是实现三民主义的最忠诚与最彻底者。

有些人怀疑共产党得势之后,是否会学俄国那样,来一个无产阶级专政及一党制度?我们的答复是:几个民主阶级联盟的新民主主义国家,和无产阶级专政的社会主义国家,是原则上不同的。中国在整个新民主主义制度期间,不可能因此就不应该是一个阶级专政与一党独占政府机构的制度,只要共产党以外的其它任何政党,任何社会集团或个人,对于共产党是采取合作的而不是采取敌对的态度,我们是没有理由不和他们合作的。俄国的历史形成了俄国的制度,在那里,废除了人剥削人的社会制度,实现了最新式民主主义即社会主义的政治、经济、文化制度,一切反对社会主义的政党都被人民抛弃了,人民仅仅拥护布尔塞维克,因此形成了俄国的局面,这在他们是完全必要与完全合理的。但是在俄国的政权机关中,即使是处在除了布尔塞维克以外没有其它政党的条件下,实行的还是工人、农民与知识分子联盟,或党与非党联盟的制度,也不是只有工人阶级或只有布尔塞维克党人才可以在政权机关中工作。中国的历史将形成中国的制度,在一个长时期中,将产生一个对于我们是完全必要与完全合理同时又区别于俄国制度的特殊形态,即几个民主阶级联盟的新民主主义的国家形态与政权形态。

在这里,也就同时回答了另一个问题。就是说,你们共产党人现在主张建立联合政府,是因为现在还没有民主选举制度,为着团结抗日,一个联合政府是需要的;但是在将来,在有了民主选举制度之后,为什么不由国民大会中的多数党组织一党政府,还要组织联合政府呢?我们回答说:中国的历史条件规定了这一点。我在前面已经说过,在中国,因为早就出现了不但代表无产阶级,而且按其纲领和实际斗争,同时也代表了最广大的农民阶级、小资产阶级、知识分子及其他民主份子的中国共产党这样一个新条件,一切情况就改变了。任何政府,如果把共产党排斥到门外,那是一件好事也做不成的,这就是中国进到了新民主主义的历史阶段的基本特点。还是孙中山先生在世时,在一九二四年,他就在实际上实现了一个有共产党参加的联合政府,从而他就做成了一件伟大的革命事业。一九二七年以后的国民党政府排斥共产党,并且举行了残酷的反共战争,这个政府所做的就变成了反革命事业了。八年抗战,虽然国民党政府直到今天,因为有一个日本侵略者站在前面,还没有向共产党公开宣布全面战争,还只用局部战争、特务镇压、封锁、诬蔑、准备内战及不许组织联合政府等方法排斥共产党,但是它已经给自己造成了这样的形势:越排斥共产党,就越走下坡路。如果今后还要排斥下去,那就是准备将下坡路走到底。在中国的条件下,一个政府里不要共产党,这是什么意思呢?这就是说,不要最广大的人民。没有人会怀疑到共产党人是为着争夺官位而要求到政府里去的,共产党人如果加入政府,这就意味着实行新民主主义的改革。中国将来有了民主选举制度以后,不论共产党是国民大会中的多数党或是少数党,政府都应该是在一个共同承认的新民主主义的纲领之下从事工作的联合政府,才有利于更好地完成新民主主义的建设事业。这一点,现在已经可以看得很清楚了。

\subsection{我们的具体纲领}

根据上述一般纲领,在各个时期中应有其具体的纲领。在整个资产阶级民主革命阶段中,在几十年中,我们的新民主主义的一般纲领是不变的。但是在这个大阶段的各个小阶段中,情形是变化了与变化着的,我们的具体纲领便不能不有所改变,这是当然的事情。例如在北伐战争时期、在土地革命时期与在抗日战争时期,我们的新民主主义的一般纲领并没有变化,但其具体纲领,三个时期中是有了变化的,这是因为我们的敌军与友军在三个时期中发生了变化的原故。

目前中国人民是处在这样的情况中:(一)日本侵略者还未被打败;(二)中国人民迫切地需要团结起来,实现一个民主的改革,以便造成民族团结,迅速地动员与统一一切抗日力量,配合同盟国打败日本侵略者;(三)国民党政府分裂民族团结,阻碍这种民主的改革。在这些情况下,我们的具体纲领即中国人民的现时要求是什么呢?

我们认为下面这些是适当的,并且是最低限度的。

中国人民要求动员一切力量,配合同盟国,彻底消灭日本侵略者,并建立国际和平;要求废止国民党一党专政,建立民主的联合政府与联合统帅部;要求惩办那些分裂民族团结与反对人民的亲日分子,法西斯主义分子与失败主义分子,造成民族团结;要求惩办那些制造内战危机的反动分子,保障国内和平;要求惩办汉奸,讨伐降敌军官,惩办日本间谍;要求取消一切镇压人民的反动的特务机关与特务活动,取消集中营;要求取消一切镇压人民的言论、出版、集会、结社思想、信仰及身体等项自由的反动法令,使人民获得充分的自由权利;要求承认一切民主党派的合法地位;要求释放一切爱国政治犯;要求撤退一切包围与进攻中国解放区的军队,并将这些军队使用于抗日前线上去;要求承认中国解放区的一切抗日军队与民选政府;要求巩固与扩大解放区及其军队,缩小沦陷区;要求帮助沦陷区人民组织地下军,准备武装起义;要求允许中国人民自动武装起来,保乡卫国;要求从政治上军事上改造那些由国民党统帅部直接领导的经常打败仗,经常压迫人民与经常排斥异己的军队,惩办那些应对溃败负责的将领;要求改善兵役制度与改善官兵生活;要求优待抗日军人家属,使前线官兵安心作战;要求优待殉国战士的遗族,优待残废军人,对于退伍军人的生活与就业,应予帮助;要求发展军事工业,以利作战;要求将同盟国的武器与财政援助公平地分配于抗战各军;要求惩办贪官污吏,实现廉洁政治;要求改善中下级公务员的待遇;要求给予中国人民以民主的自治权利;要求取消压迫人民的保甲制度;要求救济难民与救济灾荒;要求设立大量的救济基金,在国土收复后,广泛地救济沦陷区受难的人民;要求取消苛捐杂税,实行统一的累进税;要求实行农村改革,减租减息,适当地保证佃权,对贫苦农民给予低利贷款,并使农民组织起来,以利于发展农业生产;要求取缔官僚资本;要求废止现行的经济统制政策;要求制止无限制的通货膨胀与无限制的物价高涨;要求扶助民间工业,给予民间工业以借贷资本、购买原料与推销产品的便利;要求改善工人生活,救济失业工人,并使工人组织起来,以利于发展工业生产;要求取消党化教育,发展民族的科学的大众的文化教育;要求保障教职员生活及学术自由;要求保护青年、妇女、儿童的利益,救济失学青年,并使青年、妇女组织起来,以平等地位参加有益于抗日战争与社会事业的各项工作,实现婚姻自由,男女平等,实现对于青年与儿童是有益的学习;要求改善国内少数民族的待遇,允许各少数民族有民族自决权及在自愿原则下和汉族联合建立联邦国家的权利;要求保护华侨利益,扶助回国的华侨;要求保护因被日本侵略者压迫而逃来中国的外国人民,并扶助其反对日本侵略者的斗争;要求改善中苏邦交等等。而要做到这一切,最重要的是要求立即取消国民党一党专政,建立一个包括一切抗日党派及无党无派的代表人物在内的举国一致的民主的联合的临时的中央政府。没有这个前提条件,要想在全国范围内,就是说,在国民党统治区域进行稍为认真的改革,是不可能的。

只有这些,才是中国广大人民的呼声,也是同盟各国广大舆论界的呼声。

一个为各个抗日民主党派互相同意的最低限度的具体纲领,是完全必要的,我们准备拿上述纲领为基础和他们进行协商。各党可以有不同的要求,但是各党之间应该协议一个共同的要求。

这样的纲领,对于国民党统治区,暂时还是一个要求的纲领;对于沦陷区,除组织地下军准备武装起义一项外,是一个等到收复后才能实施的纲领;对于解放区,则是一个早已实施并正在继续实施的纲领。

\subsubsection{彻底消灭日本侵略者不许中途妥协}

在上述中国人民的目前要求,或具体纲领中,包含着许多战时战后的重大问题,需要在下面加以说明。在说明这些问题时,我们将批评国民党主要统治集团的一些错误观点,同时也将回答其他人们的一些疑问。

第一,彻底消灭日本侵略者,不许中途妥协。

开罗会议决定应使日本侵略者无条件投降,这是正确的。但是,现在日本侵略者正在暗地里进行活动,企图获得妥协的和平,国民党政府中的亲日分子,经过南京傀儡政府,亦正在和日本密使勾勾搭搭,并未遇到制止,因此,中途妥协的危险并未完全过去。开罗会议又决定将东北四省、台湾、澎湖群岛归还中国,这是很好的。但是根据国民党政府的现行政策,要想依靠它去打到鸭绿江边,收复一切失地,是不可能的。在此种情形下,中国人民应该怎么办呢?中国人民应该要求国民党政府彻底消灭日本侵略者,不许中途妥协。一切妥协的阴谋活动必须立刻制止。中国人民应该要求国民党政府改变现在的消极的抗日政策,将其一切军事力量用于积极作战。中国人民应该扩大自己的军队——八路军、新四军及其他人民军队,并在一切敌人所到之处,广泛地自动地发展抗日武装,准备直接配合同盟国作战,收复一切失地,决不要单纯依靠国民党。消灭日本侵略者是中国人民的神圣权利。任何反动分子,要想剥夺中国人民的这种神圣权利,要想压制中国人民的抗日活动,要想破坏中国人民的抗日力量,中国人民在其劝说无效之后,应该站在自卫立场上给以坚决的回击。因为中国反动分子的这种背叛民族利益的反动行为,完全是帮助日本侵略者的。

\subsubsection{废止一党专政,建立联合政府}

第二,废止国民党一党专政,建立民主的联合政府。

为着彻底消灭日本侵略者,必须在全国范围内实行民主改革。而要这样做,不废止国民党一党专政,建立民主的联合政府,是不可能的。

所谓国民党一党专政,实际上是国民党内反人民集团的专政,它是中国民族团结的破坏者,是抗日失败的负责者,是动员与统一中国人民抗日力量的根本障碍物,八年抗日战争中的惨痛经验,中国人民已经深刻地认识了这一点,很自然地要求立即废止它。这个反人民的专政,又是内战的祸胎,如不立即废止,内战惨祸又将降临。

中国人民要求废止这个反人民专政的呼声是如此普遍而响亮了,使得国民党当局自己也不能不公开承认“提早结束训政”,可见这个所谓“训政”,或“一党专政”之丧失人心,威信扫地,到了何种地步了。在中国,已经没有一个人还敢说“训政”或“一党专政”有什么好处,不应该废止或结束了,这是当前时局的一大变化。

应该结束是确定了,毫无疑义了。但是如何结束呢?可就意见纷歧了。一个说:立即结束,成立民主的临时的联合政府。一个说:等一会再结束,召开“国民大会”,“还政于民”,却不能还政于联合政府。

这是什么意思呢?

这是两种做法的表现:真做与假做。

第一种,真做。这就是立即宣布废止国民党一党专政,成立一个由国民党、共产党、民主同盟及无党无派分子的代表人物联合组成的临时的中央政府,发布一个民主的施政纲领,如同我们在前面提出的那些“中国人民的现时要求”以便恢复民族团结,打败日本侵略者。为着讨论这些事情,召集一个各党各派及无党无派的代表人物的圆桌会议,成立协议,动手去做。这是一个团结的方针,中国人民是坚决拥护这个方针的。

第二种,假做。不顾广大人民及一切民主党派的要求,一意孤行地召开一个由国民党反人民集团一手包办的所谓“国民大会”,在这个会上通过一个实际上维持独裁反对民主的所谓“宪法”,使那个仅仅由几十个国民党人私自委任的完全没有民意基础的强安在人民头上的不合法的所谓“国民”政府,披上“合法”的外衣,装模作样地“还政于民”,实际上,依然是“还政”于国民党内的反人民集团。谁要不赞成,就说他是破坏“民主”,破坏“统一”,就有“理由”向他宣布讨伐令。这是一个分裂的方针,中国人民是坚决反对这个方针的。

我们的反人民的英雄们根据这种分裂方针所准备采取的步骤,有把他们自己推到绝路上去的危险性。他们准备把一条绳索套在自己的脖子上,并且让它永远也解不开,这条绳索的名称就叫做“国民大会”。他们的原意是想把所谓国民大会当作法宝,祭起来。一则抵制联合政府,二则维持独裁统治,三则准备内战理由的,可是历史的逻辑将向他们所设想的反面走去:“搬起石头打自己的脚”。因为现在谁也明白,在国民党统治区域,人民没有自由,在日本人占领区域,人民不能参加选举,在有了自由的中国解放区,国民党政府又不承认它,在此种情况下,那里来的国民代表?那里来的国民大会?现在叫着要开的,是那个还在内战时期,还在八年以前,由国民党独裁政府一手伪造的所谓国民大会,如果这个会开成了,势必闹到全国人民群起反对,请问我们的反人民英雄们如何下台?归根结底,伪造国民大会如果开成了,不过将他们自己推到绝路上,造成分崩离析的局面。

我们共产党人是不想中国走上这个局面的,所以提出解救中国的两个步骤:第一个,目前时期,经过各党各派及无党无派代表人物的协议,成立临时的联合政府;第二个,将来时期,经过自由的无拘束的选举,召开国民大会,成立正式的联合政府。总之,都是联合政府,团结一切愿意参加的阶级与政党的代表在一起,在一个民主的共同纲领之下,为现在的抗日与将来的建国而奋斗。

不管国民党人或任何其他党派、集团及个人如何设想,愿意或不愿意,自觉或不自觉,中国只能走这条路。这是一个必然的、不可避免的历史法则,任何力量,都是扭转不过来的。

在这个问题及其他任何有关民主改革的问题上,我们共产党人声言:不管国民党当局现在还是怎样坚持其错误政策与怎样借谈判为拖延时间、搪塞舆论的手段,只要他们一旦愿意放弃其错误的现行政策,同意民主改革,我们是愿意和他们恢复谈判的。但是谈判的基础必须放在抗日、团结与民主的总方针上,一切离开这个总方针的所谓办法、方案,或其他空话,不管它怎样说得好听,我们是不能赞成的。

\subsubsection{人民的自由}

第三,人民的自由。

目前中国人民争自由的目标,首先地与主要地是向着日本侵略者。但是国民党政府剥夺人民的自由,捆起人民的手足,使他们不能反对日本侵略者,不解决这个问题,就不能在全国范围内动员与统一一切抗日力量。我们在纲领中提出了废止一党专政,成立联合政府,取消特务,取消镇压自由的法令,惩办汉奸、间谍、亲日分子、法西斯分子与贪官污吏,释放政治犯,承认各民主党派的合法地位,撤退包围与进攻解放区的军队,承认解放区,废止保甲制度,以及其他许多经济的文化的与民众运动的要求,就是为着解开套在人民身上的绳索,使人民获得抗日、团结与民主的自由。

自由是人民争来的,不是什么人恩赐的。中国解放区的人民已经争得了自由。其他地方的人民也可能与应该争得这种自由。中国人民争得的自由越多,有组织的民主力量越大,一个统一的临时的联合政府便越有成立之可能。这种联合政府一经成立,它将转过来给予人民以充分的自由,巩固联合政府的基础。然后才有可能,在日本侵略者被消灭之后,在全部国土上进行自由的无拘束的选举,产生民主的国民大会,成立统一的正式的联合政府。没有人民的自由,就没有真正民选的国民大会,就没有真正民选的政府。难道还不清楚么?

人民的言论、出版、集会、结社、思想、信仰与身体这几项自由,是最重要的自由。在中国境内,只有解放区是彻底地实现了。

一九二五年,孙中山先生在其临终的遗嘱上说:“余致力国民革命凡四十年,其目的在求中国之自由平等。积四十年之经验,深知欲达到此目的,必须唤起民众及联合世界上以平等待我之民族,共同奋斗。”背叛孙先生的不肖子孙,不是唤起民众,而是压迫民众,将民众的言论、出版、集会、结社、思想、信仰与身体等项自由权利剥夺干净。对于认真唤起民众、认真保护民众自由权利的共产党、八路军、新四军及解放区,则称之为“奸党”、“奸军”、“奸区”。我们希望这种颠倒是非的时代快要过去了。再要延长这种颠倒,中国人民将不能忍耐了。

\subsubsection{人民的统一}

第四,人民的统一。

为着消灭日本侵略者,为着防止内战,为着建设新中国,必须将分裂的中国变为统一的中国,这是中国人民的历史任务。

但是如何统一呢?独裁者的专制的统一,还是人民的民主的统一?从袁世凯以来,北洋军阀强调专制统一。但是结果怎么样呢?和这些军阀的志愿相反,所得的不是统一而是分裂,最后是自己从台上滚下去。国民党反人民集团抄袭袁世凯老路,追求专制的统一,打了整十年内战,结果把一个日本侵略者打了进来,自己也缩上了峨嵋山。现在又在山上大叫其专制统一论,这是叫给谁听呢?难道还有什么爱国的有良心的中国人愿听么?经过了十六年的北洋军阀统治,又经过了十八年的国民党独裁统治,人民有了充分的经验,有了明亮的眼睛,他们要一个人民大众的民主的统一,不要独裁者的专制的统一。我们共产党人还在一九三五年就提出抗日民族统一战线的方针,没有一天不为此而奋斗。一九三九年国民党推行其反动的“限制异党活动办法”,发生投降、分裂、倒退的危机,国民党人大叫其专制统一论时,我们又说:“非统一于投降而统一于抗战,非统一于分裂而统一于团结,非统一于倒退而统一于进步,只有这后一种统一才是真统一,其它一切都是假统一。”又过了六年了,问题还是一样。

没有人民的自由,没有人民的民主政治,能够统一么?有了这些,立刻就统一了。中国人民争自由,争民主,争联合政府的运动,就是争统一的运动。我们在具体纲领中提出了许多争自由争民主的要求,提出了联合政府的要求,就是为了这个目的。不废止国民党内反人民集团的专政,成立民主的联合政府,不但在国民党统治区不能实行任何民主的改革,不能动员那里的全体军民有力地配合同盟国彻底消灭日本侵略者,而且还将发展为内战的惨祸,这是很多人都明白的常识了。为什么如此众多的有党有派无党无派的民主分子,包括国民党内的许多民主分子在内,一致要求联合政府,就是看清楚了时局的危机,非如此不能克服这种危机,不能达到团结对敌与团结建国之目的。

\subsubsection{人民的军队}

第五,人民的军队。

中国人民要自由,要统一,要联合政府,要彻底消灭日本侵略者与建设新中国,没有一枝站在人民立场上的军队,那是不行的。彻底站在人民立场的军队,现在还只有一个八路军、新四军,还很不够,可是国民党内的反人民集团却处心积虑地要破坏与消灭这枝军队。一九四四年,国民党政府提出了一个所谓“提示案”,叫共产党“限期取消”解放区军队的五分之四。一九四五年,即最近的一次谈判,又叫共产党将解放区军队全部交给它,然后它给共产党以“合法地位”。

这些人们向共产党人说:你交出军队,我给你自由。根据这个学说,没有军队的党派该有自由了。但是一九二四至一九二七年,中国共产党只有很少一点军队,国民党政府的“清党”政策与屠杀政策一来,自由也光了。现在的中国民主同盟,中国国民党民主派并没有军队,同时也没有自由。十八年中,在国民党政府统治下的工人、农民、学生及一切要求进步的文化界、教育界、产业界,他们也一概没有自由。难道是由于他们组织了一枝什么军队,实行了什么“封建割据”,成立了什么“奸区”,违反了什么“政令军令”,因此才不给自由的么?否,适得其反,正是因为他们没有这样做。

“军队是国家的”,非常之正确,世界上没有一个军队不是属于国家的。但是什么国家呢?大地主,大银行家、大买办的封建法西斯独裁的国家?还是人民大众的新民主主义的国家?中国只应该建立新民主主义的国家,并在这个国家基础之上建立新民主主义的联合政府,中国的一切军队均应属于这个国家的这个政府,借以保障人民的自由,有效地反对外国侵略者。什么时候中国有一个新民主主义的联合政府与联合统帅部出现了,中国解放区的军队将立即交给它。但是一切国民党的军队也必须同时交给它。

为着打败日本侵略者与建设新中国,为着中国的自由与统一,为着防止内战,保障国内和平,中国人民必须做一项责无旁贷的工作,这就是将国民党政府的那些在对日作战时经常打败仗的,以反对人民,排斥异己与准备内战为目的的军队,加以改造,变为人民的军队。国民党军队的士兵与广大数量的军官原来是好的,他们对抗日原来是积极的,曾经英勇作战过。他们也并不愿意反对共产党、八路军、新四军与解放区。仅因国民党统帅机关及腐败将领违反孙中山先生在世时所手创的民主传统,实行自己的失败主义与法西斯主义的领导,反动的政治工作与反动的特务网,迫使这个军队站在反对人民的立场上;并使这个军队处于严重情况之中,内而官兵关系,外而军民关系,都极端恶化,战斗力薄弱,生活痛苦,兵役制度百弊丛生,又不许农村人民武装起来保乡卫国。一切这些,均须改革,决不许可再这样长期继续下去。这种改革,不但是中国人民的要求,也是同盟国广大舆论界的要求,也是国民党军队内部全体士兵与广大军官们的要求。

一九二四年,孙中山先生说:“今日以后,当划一国民革命之新时代。……第一步,使武力与国民相结合。第二步,使武力为国民之武力。”八路军、新四军遵循这个方针,成了“国民之武力”,就是说,成了人民的军队,所以能打胜仗。国民党军队在北伐战争前期,做到了孙先生所说的“第一步”,所以打了胜仗。从北伐后期直至现在,连“第一步”也丢了,站在反人民的立场上,所以一天一天腐败堕落,除了“内战内行”之外,对于“外战”,就不能不是一个“外行”。国民党军队中一切爱国的有良心的军官们,应该起来恢复孙先生的精神,改造自己的军队。

在改造旧军队的工作中,对于一切可以教育的军官,应当给予适当的教育,帮助他们学得正确观点,清除陈旧观点,为人民的军队而继续服务。

为创造中国人民的军队而奋斗,是全国人民及一切民主党派的责任。没有一个人民的军队,便没有人民的一切。对于这个问题,切不可只发空论。

我们共产党人愿意赞助改革中国军队的事业。八路军、新四军对于一切愿意团结人民、反对日本侵略者、而不反对中国解放区的军队,均应看作自己的友军,给以适当的协助。

\subsubsection{土地}

第六,土地问题。

为着消灭日本侵略者与建设新中国,必须实行土地改革、解放农民。孙中山先生的“耕者有其田”的主张,是目前资产阶级民主主义性质的革命时代的正确主张。

为什么把目前时代的革命叫做“资产阶级民主主义性质的革命”?这就是说,这个革命的对象不是一般的资产阶级,而是民族压迫与封建压迫;这个革命的一切设施,不是一般地废除私有财产,而是一般地保护私有财产;这个革命的结果,将为资本主义扫清道路而使之获得发展。“耕者有其田”,是把土地从封建剥削者手里转移到农民手里,变为农民的私有财产,使农民从封建的土地关系上获得解放,使农业从旧式的落后的水平进到近代化的水平,从而使工业获得市场,造成了将农业国转变为工业国的可能性。因此,“耕者有其田”的主张,是一种资产阶级民主主义性质的主张,并不是无产阶级社会主义性质的主张,是一切革命民主派的主张,并不单是我们共产党人的主张。所不同的,在中国条件下,只有我们共产党人把这项主张看得特别认真,不但口讲,而且实做。那些人们是革命民主派呢?除了无产阶级是最彻底的革命民主派之外,农民是最大的革命民主派。农民的绝对大多数,就是说,除开那些带上了封建尾巴的一部分富农之外,无不积极地要求“耕者有其田”。城市小资产阶级也是革命民主派,“耕者有其田”使农业生产力获得发展,对于他们是有利的。自由资产阶级是一个动摇的阶级,他们需要市场,他们也赞成“耕者有其田”,他们多半和土地联系着,他们中的许多人又惧怕“耕者有其田”。坚决反对“耕者有其田”的是国民党内的反人民集团,因为他们是代表大地主、大银行家、大买办阶层的。中国没有单独代表农民的政党。自由资产阶级的政党没有坚决的土地纲领。因此,只有具备了坚决的土地纲领,为农民利益而认真奋斗,因而获得最广大农民群众作为自己伟大同盟军的中国共产党人,成了农民与一切革命民主派的领导者。

一九二七至一九三六年,中国共产党执行了孙先生“耕者有其田”的主张。出而张牙舞爪,进行了十年反人民战争,亦即反“耕者有其田”的战争的,就是那个集中了孙中山先生一切不肖子孙在内的团体——国民党内的反人民集团。

抗战期间,中国共产党让了一大步,将“耕者有其田”的政策,改为减租减息政策。这个让步是正确的,推动了国民党参加抗日,又使解放区的地主与农民联合起来反对日本侵略者。这个政策,如果没有特殊阻碍,我们准备在战后继续下去,首先在全国实现减租减息,然后寻找适当方法,有步骤地达到“耕者有其田”。

但是背叛孙先生的人们不但反对“耕者有其田”,连减租减息也反对。国民党政府自己颁布的“二五减租”一类法令,自己不实行,仅仅在中国解放区实行了,因此也就成立了罪状:名之日“奸区”。

在抗战期间,出现了所谓民族阶段与民主民生阶段的两阶段论,这是错误的。

大敌当前,民主民生不该提起,等日本人走了再提好了——这是国民党反人民集团的谬论,其目的是不愿抗日战争获得彻底胜利,有些人居然随声附和,和自觉地作了这种谬论的尾巴。

大敌当前,不解决民主民生就不能赶走日本人——这是中国革命民主派的正论,为整个中国近代史,尤其是八年抗战史所证明,也为法兰西、意大利、波兰、南斯拉夫、保加利亚、罗马尼亚、匈牙利等国人民的反法西斯斗争所证明,在波兰等国所执行的,不是如我们的减租减息,而是“耕者有其田”。

在抗日期间,一切服从抗日,减租减息及一切民主改革是为着抗日的,不是自己孤立起来的。为着团结一切社会阶层反对共同敌人,不取消地主的土地所有权,并适当地交租交息,奖励地主的资本向工业方面转移,同时,更使开明绅士和其他人民的代表一道参加社会工作与政府工作。对于富农,则奖励其发展生产。所有这些,是在坚决执行农村民主改革的路线里包含着的,是完全必要的。

两条路线:或者坚决反对中国农民解决民主民生问题,而使自己腐败无能,无力抗日。或者坚决赞助中国农民解决民主民生问题,而使自己获得占全人口百分之八十的最伟大的同盟军,借以组织雄厚的战斗力量。前者就是国民党政府的路线,后者就是中国解放区的路线。

动摇于两者之间,口称赞助农民,但不坚决实行减租减息,武装农民与建立农村民主政权,这是机会主义者的路线。

国民党反人民集团动员一切力量,向着中国共产党放出了一切恶毒的箭:明的和暗的,军事的和政治的,流血的和不流血的。两党的争论,就其社会性质说来,实质上是在农村关系的问题上。我们究竟在那一点上触怒了国民党反人民集团呢?难道不正是在这个问题上面吗?国民党反人民集团之所以受到日本侵略者的欢迎与鼓励,难道不正是在这个问题上面,给日本侵略者帮了大忙吗?所谓“共产党破坏抗战、危害国家”,所谓“奸党”、“奸军”、“奸区”,所谓“不服从政令、军令”,难道不正是因为中国共产党在这个问题上做了真正符合于民族利益的认真的事业吗?

农民——这是中国工人的前身。将来还要有几千万农民进人城市,进入工厂。如果中国需要建设强大的民族工业,建设很多的近代式的大城市,就要有一个变农村人口为城市人口的长过程。

农民——这是中国工业的市场。只有他们能够供给最丰富的粮食、原料与吸收最广大的工业品。

农民——这是军队的来源。士兵就是穿起军服的农民,他们是日本侵略者的死敌。

农民——这是现阶段中国民主政治的主要基础。中国的民主主义者如不依靠三万万六千万农民群众的援助,他们就将一事无成。

农民——这是现阶段中国文化运动的主要基础。所谓扫除文盲,所谓普及教育,所谓大众文艺,所谓国民卫生,离开了三万万六千万农民,岂非大半成了空话?

我说的“主要基础”,当然不是忽视其他约占人口九千万的人民在政治上经济上文化上的重要性,尤其不是忽视在中国人民中政治上最觉悟与具有领导一切民主运动资格的工人阶级,这是不应该发生误会的。

认识这一切,不但中国共产党人,而且一切民主派,都是完全必要的。

土地制度获得改革,甚至仅获得初步的改革,例如减租减息之后,农民的生产兴趣就增加了。然后帮助农民在自愿原则下,逐渐地组织在农业生产合作社及其它合作社之中,生产力就会发展起来。这种农业生产合作社,现时还只能是建立在农民个体经济基础上的(农民私有财产基础上的)集体的互助的劳动组织,例如变工队、互助组、换工班之类,但是生产力的发展与生产量的增加,已属惊人。这种制度,已成为中国解放区的普遍的制度,今后应当尽量推广。

这里应当指出一点,就是说,变工队一类的合作组织,原来在农民中就有了的,但在那时,不过是农民悲惨生活的表现。现在中国解放区的变工队,形式与内容都起了变化,它是农民群众为着发展自己的生产力,争取富裕生活的表现。

中国一切政党的政策及其实践在中国人民中所表现的作用的好、坏、大、小,归根到底,看其对于中国人民的生产力的发展是否有帮助及其帮助之大小。它是束缚生产力的?还是解放生产力的?彻底消灭日本侵略者,实行土地改革,解放农民,发展现代工业,建立独立、自由、民主、统一与富强的新中国,只有这一切,才能使中国社会生产力获得解放,才是中国人民所欢迎的。

这里还要指出一点,就是说,从城市到农村工作的知识分子,不容易了解现时还是分散的落后的个体经济这种农村特点,在解放区,则还要加上暂时还是被敌人分割的与游击战争的这些特点。因为不了解这些特点,他们就往往不适当地带着他们在城市里生活或工作的观点去观察农村问题,去处理农村工作,因而脱离农村实际,不能和农民打成一片。这种现象,必须用教育方法加以克服。

中国广大的革命知识分子应该觉悟到将自己和农民结合起来之必要。农民正需要他们,等待他们的援助。他们应该热情地跑到农村中去,脱下学生装,穿起粗布衣,不惜从任何小事情做起,在那里了解农民的要求,帮助农民觉悟起来,组织起来,为着完成中国民主革命中一项极重要工作即农村民主革命而奋斗。

在日本侵略者被消灭之后,日本侵略者及重要汉奸分子的土地应该被没收,并分配给无地及少地的农民。

\subsubsection{工业}

第七,工业问题。

为着打败日本侵略者与建设新中国,必须发展工业。但是,在国民党政府领导之下,一切依赖外国,自己的财政经济政策是破坏一切经济生活的,国民党统治区内仅有的一点小型工业,也不能不处于大部分破产的状态中。政治不改革,一切生产力都遇着破坏的命运,农业如此,工业也是如此。

就整个来说,没有一个独立、自由、民主与统一的中国,不能有工业的中国。消灭日本侵略者,这是谋独立。废止国民党一党专政,成立民主的联合政府,实现人民的自由,人民的统一,人民的军队,实现土地改革,解放农民,这是谋自由、民主与统一。没有独立、自由、民主与统一,不可能有真正大规模的全国性的工业。没有工业,便没有巩固的国防,没有人民的福利,没有国家的富强。一八四〇年鸦片战争以来一百零五年的历史,特别是国民党当政以来十八年的历史,清楚地把这个要点告诉了中国人民。一个不是贫弱的而是富强的中国,是和一个不是殖民地半殖民地的而是独立的,不是半封建的而是自由的,民主的,不是分裂的而是统一的中国,相联结的。在一个半殖民地的、半封建的、分裂的中国里,要想发展工业,建设国防,福利人民,招致国家的富强,多少年来多少人做过这种梦,但是一概幻灭了。许多好心的教育家、科学家、学生们不问政治,自以为可以所学为国家服务,结果也化成了梦,一概幻灭了。这是好消息,这种幼稚的梦的幻灭,正是中国富强的起点。中国人民在抗日战争中学得了许多东西,知道在日本侵略者被消灭之后,有建立一个新民主主义的独立、自由、民主、统一、富强的中国之必要,而这些条件是互相关联的,不可缺一的。果然如此,中国就有希望了。解放中国人民的生产力,使之获得充分发展的可能性,有待于新民主主义的政治条件在全中国境内的实现。这一点,懂得的人是一天一天的多起来了。

在新民主主义的政治条件获得之后,中国人民及其政府必须采取切实的步骤,在若干年内逐步地建立轻重工业,使中国由农业国地位升到工业国地位上去。中国的新民主主义的独立、自由、民主与统一,如无巩固的经济做它们的基础,如无进步的比较现时发达得多倍的农业,如无大规模的在全国经济比重上占极大优势的工业以及与此相适应的交通、贸易、金融等事业做它们的基础,所谓新民主主义的独立、自由、民主与统一,是不能巩固的。

我们共产党人愿意协同全国一切抗日民主党派,各部分产业界,为上述目标而奋斗。中国工人阶级在这个任务中将起伟大的作用。

中国工人阶级,自第一次世界大战以来,就开始以自觉的姿态,为中国的独立、解放而斗争。一九二一年,产生了它的先锋队——中国共产党。从此以后,使中国的解放斗争进入了新阶段。在北伐战争、土地革命与抗日战争三个时期中,中国工人阶级及其先锋队,对于中国人民的解放事业,作了极大的努力与极有价值的贡献。在最后消灭日本侵略者的斗争中,特别是在收复大城市与交通要道的斗争中,中国工人阶级将起着极大的作用。在抗日结束以后,可以预断,中国工人阶级的努力与贡献将会是更大的。中国工人阶级的任务,不但是为着中国的独立、自由、民主与统一而斗争,而且是为着中国的工业化与农业近代化而斗争。

在新民主主义的国家制度下,将采取调节劳资间利害关系的政策。一方面,保护工人利益,根据情况之不同,而实行八小时到十小时的工作制以及适当的失业救济,社会保险,工会的权利等;另一万面,保证国家企业、私人企业与合作社企业在合理经营下的正当赢利。总之,使劳资双方共同为发展工业生产而努力。

为着发展工业,需要大批资本。从什么地方来呢?不外两方面:主要地依靠中国人民自己积累资本,同时借助于外援。在服从中国法令,有益中国经济的条件之下,外国投资是我们所欢迎的。对于中国人民与外国人民都有利的事业,是中国在得到一个巩固的国内和平与国际和平,得到一个彻底的政治改革与土地改革之后,能够蓬蓬勃勃地发展大规模的轻重工业与近代化的农业。在这个基础上,外国投资的容纳量将是产常广大的。一个政治上倒退与经济上贫困的中国,则不但对于中国人民非常不利,对于外国人民也是不利的。

日本侵略者被消灭之后,日本侵略者及重要汉奸分子的企业及财产,应当没收,归政府处理。

\subsubsection{文化、教育、知识分子}

第八,文化、教育、知识分子问题。

民族压迫与封建压迫所给予中国人民的灾难中,包括了民族文化的灾难。特别是具有进步意义的文化与教育事业,文化人与教育家,所受灾难,更为深重。

为着扫除民族压迫与封建压迫,为着建立新民主主义的独立、自由、民主、统一与富强的中国,需要大批的人民的教育家,教师,人民的科学家,工程师,技师,医生,新闻工作者,著作家,文学家,艺术家与普通文化工作者,以“为人民服务”“和人民打成一片”的精神,从事艰巨的工作。一切这些知识分子,只要是在为人民服务中著有成绩的,应受到政府与社会的尊重,把他们看作国家与社会的宝贵财富。中国是一个被民族压迫与封建压迫所造成的文化落后的国家,中国的人民解放斗争迫切地需要知识分子,因而知识分子问题就特别显得重要。而在过去半世纪的人民解放斗争,特别是五四运动以来的斗争中,在八年抗日中,广大革命知识分子对于中国人民解放事业所起的作用,是很伟大的,在今后的斗争中,他们将起更大的作用。因此,今后政府应有计划地从广大人民中培养各类知识分子干部,并注意团结与教育现有一切有用的知识分子。

从百分之八十的人口中扫除文盲,是建立新中国的必要条件。

一切奴化的、封建主义的与法西斯主义的文化、教育,应当采取适当的但是坚决的步骤,加以扫除。

对于由民族压迫与封建压迫所造成的摧残中国人民的精神与肉体的那种不知卫生的愚昧与疾病疫疠的严重情况,应当讲求积极的改革与救治办法,推广国民卫生事业。

对于旧文化工作者,旧教育工作者及旧医生们的态度,是采取适当方法,教育他们,使他们获得新观点,新方法,为中国人民服务。

中国国民文化与国民教育的宗旨,应当是新民主主义的,就是说,中国应当建立自己的民族的、科学的、人民大众的新文化的新教育。

对于外国文化,排外主义的方针是错误的,应当尽量吸收进步的外国文化,以为中国文化运动的借镜。盲目服从的方针也是错误的,应当以中国人民的实际需要为基础,批判地吸收外国文化。对于中国古代文化,同样,既不是一概排斥,也不是盲目服从,而是批判地接收它,以利于推进中国的新民主主义的文化。

在为着中国人民解放事业而斗争的总方针下,共产党人应当不分阶级、信仰与党派,和一切知识分子很好地团结起来。

\subsubsection{少数民族}

第九,少数民族问题。

国民党反人民集团否认中国有多民族存在,而把蒙、回、藏、彝(“夷」)、苗、瑶(「猺”)各少数民族称之为“宗族”。他们对于各少数民族,完全继承满清政府及北洋军阀政府的反动政策,压迫剥削,无所不至。一九四三年对于伊克昭盟蒙族人民的屠杀事件,一九四四年直至现在对于新疆少数民族的武力镇压事件以及近几年对于甘肃回民的屠杀事件,就是明证。这是法西斯主义的大汉族主义的错误的民族思想与错误的民族政策,完全背叛了孙中山先生。

一九二四年,孙中山先生在其所著的中国国民党第一次代表大会宣言里说:

“国民党之民族主义,有两方面之意义:一则中国民族自求解放;二则中国境内各民族一律平等。”“国民党敢郑重宣言:承认中国以内各民族之自决权;于反对帝国主义及军阀之战争获得胜利以后,当组织一自由统一的(各民族自由联合的)中华民国。”

中国共产党完全同意上述孙先生的民族政策,共产党人应当积极地帮助各少数民族的广大人民群众为实现这个政策而奋斗。应当帮助各少数民族的广大人民群众,包括一切联系民众的领袖在内,争取他们在政治上、经济上、文化上的解放与发展,并成立拥护民众利益的少数民族自己的军队。他们的言语、文字、风俗、习惯及宗教信仰,应被尊重。

多年以来,陕甘宁边区及华北解放区对待蒙回两族的态度是正确的,其工作是有成绩的。

\subsubsection{外交}

第十,外交问题。

中国共产党同意大西洋宪章及莫斯科、开罗、德黑兰、克里米亚各次国际会议的决议,因为这些国际会议的决议都是有利于打败法西斯侵略者与维持世界和平的。

中国共产党尤其对于克里米亚会议的决议表示热烈的赞同,克里米亚会议决定:最后地打败法西斯德国,并消灭法西斯主义及其产生的原因;在欧洲解放区消灭法西期主义的最后残余,确立各国的国内和平,建立各国人民自己所选择的民主制度;而在建立民主制度的程序方面:“组织临时政权机关,这种政权机关广泛地包罗人口中一切民主成分的代表;并保证尽早经过自由选举,以建立执行人民意志的政府。”克里米亚会议并决定:英美苏三国团结一致,保持“巩固的与持久的”世界和平,迅速成立世界和平机构。

我们认为克里米亚的路线,和中国共产党关于解决东方问题与中国问题的基本方针,是一致的。在消灭日本侵略者与解决中国问题时,下面各点是必须实现的。第一,日本侵略者必须最后地被打败,并彻底消灭日本法西斯主义、军国主义及其产生的原因,不许中途妥协。第二,中国法西斯主义的最后残余必须被消灭,不许保留丝毫。第三,中国必须建立国内和平,不许再打内战。第四,国民党独裁统治必须废止;废止之后,首先代之以广泛地包罗中国人口中一切民主成份的代表所组成的举国一致的临时的联合政府,这是第一个步骤;然后,在国土收复之后,经过自由的无拘束的选举,建立执行人民意志的正式的联合政府,这是第二个步骤;这两个步骤,不许可省掉任何一个。按照克里米亚路线,按照中国国情,我们必须这样做。

中国共产党的外交政策的基本原则,是在彻底消灭日本侵略者,保持世界和平,互相尊重国家的独立与平等地位,互相增进国家与人民的利益及友谊这些基础之上,和各国建立和巩固邦交,解决一切战时战后的相互关系问题,例如配合作战、和平会议、通商、投资等等。

中国共产党对于保障战后国际和平与安全的机构之建立,完全同意敦巴顿橡树林会议所作的建议及在克里米亚会议上对此问题所作的决定。中国共产党欢迎旧金山联合国代表大会。中国共产党已经派遣自己的代表加入中国代表团去出席旧金山会议,借以表达中国人民的意志。

我们认为国民党政府必须停止对于苏联的仇视态度,迅速地改善中苏邦交。苏联是第一个废除不平等条约,并与中国订立平等新约的国家。在一九二四年孙中山先生亲自召集的第一次国民党代表大会开会时及在其后举行北伐战争时,苏联是当时唯一援助中国解放战争的国家。在一九三七年七七抗战以后,苏联又是第一个援助中国反对日本侵略者的国家。中国人民对于苏联政府与苏联人民的这些援助,表示感谢。我们认为太平洋问题的最后的彻底的解决,没有苏联参加是不可能的。

我们认为英美两大国,特别是美国,在反对日本侵略者的共同事业上所作的伟大努力,以及两国政府与两国人民对于中国的同情与援助,是值得感谢的。

但是,我们要求各同盟国政府,首先是英美两国政府,对于中国最广大人民的呼声,加以严重的注意,不要使他们自己的外交政策违反中国人民的意志,因而损害或失去中国人民的友谊。我们认为任何外国政府,如果援助中国反动分子进行反对中国人民的民主事业,那就将要犯下绝大的错误。

中国人民欢迎许多外国政府宣布废除对于中国的不平等条约、并与中国订立平等新约这种以平等地位对待中国人民的措施。但是,我们认为平等条约的订立,并不就是中国在实际上已经取得真正的平等地位。这种实际上的真正的平等地位,决不能单靠外国政府与外国人民的好意给与,主要地应靠中国人民自己努力,把中国在政治上经济上文化上建设成为一个新民主主义的独立、自由、民主、统一与富强的国家,否则便只会有形式上的独立、平等,在实际上是不会有的。就是说,依据国民党政府的现行政策,决不会使中国获得真正的独立与平等。

我们认为在日本侵略者被打败并无条件投降之后,为着彻底消灭日本法西斯主义、军国主义及其所由产生的政治、经济、社会的原因,必须帮助一切日本人民的民主力量,建立日本人民的民主制度。没有这种日本人民的民主制度,便不能彻底消灭日本法西斯主义与军国主义,便不能保证太平洋的和平。

我们认为开罗会议给予朝鲜独立的决定是正确的,中国人民应当帮助朝鲜人民获得解放。

美国已经允许菲律宾独立。我们希望英国也能允许印度独立。因为一个独立的民主的印度,不但是印度人民的需要,也是世界和平的需要。

对于南洋各国——缅甸、马来亚、荷属东印度、法属安南,我们希望英、美、法、荷诸国于帮助当地人民打败日本侵略者以后,能够仿照克里米亚会议对于欧洲解放区所取的态度,给予当地人民以建立独立的民主的国家制度之权利。对于泰国,则仿照对待欧洲法西斯附属国的方法去处理。

已故的美国总统罗斯福先生说:“世界已经缩小了。”的确是这样,对于中国人民曾经感觉是住在十分遥远地方的美国人民,现在感觉成了近邻了。中国人民将和美英苏法各大国的人民,以及全世界上一切国家的人民一道,共同建设一个“巩固的与持久的”世界和平。

关于具体纲领的说明,主要的就是这样。

再说一遍,一切这些具体纲领,如果没有一个举国一致的民主的联合政府,就不可能顺利地在全中国实现。

中国共产党在其为中国人民的解放事业而奋斗的二十四年中,创造了这样的地位,就是说,不论什么政党或社会集团,也不论是中国人或外国人,如果采取抹杀或不尊重中国共产党的意见的态度,那是极不妥当的。过去和现在都有这样的人,企图孤行己见,敢于抹杀或不尊重我们的意见,但是所得的结果只有一个,就是行不通。这是什么原故呢?不是别的,就是因为我们的意见,我们的政策,我们在中国现阶段上所提出与所实施的新民主主义的一般纲领与具体纲领,符合于最广大的中国人民的利益。我们是中国人民的最忠实的代言人,谁要是敢于抹杀或不尊重我们,谁就是在实际上抹杀或不尊重最广大的中国人民,谁就难免要失败。

\subsection{中国国民党统治区的任务}

关于我党的新民主主义的一般纲领与具体纲领,我已在上面作了充分的说明。无疑地,这些纲领,是要在全中国实行的,整个国际国内的形势,给中国人民展开了这种想望。但是,目前的情形,不能不使我们在实行时有所区别,这就是在国民党统治区、在沦陷区、在解放区三种地方互不相同。根据三种地方的不同的情形,产生了不同的任务。这些任务,有些我已经在前面说到了,有些还须在下面加以补充。

在国民党统治区,人民没有爱国活动的自由,民主运动被认为非法,但是包括许多阶层、许多民主党派与民主分子的积极活动是在发展中。中国民主同盟,在今年一月发表了要求结束国民党一党专政与成立联合政权的宣言。社会各界发表同类性质的宣言的,还有许多。国民党内相当广大的党员群众及许多重要人物,对于他们自己党的领导机关的政策,日益表示怀疑和不满,日益感觉他们自己的党从广大中国人民孤立起来的危险性,而要求有一种适合时宜的民主的改革。以重庆为中心的工人、农民、公教人员、商人、产业界、文化界、学生界、教育界、妇女界,乃至一部分军人的民主运动,正在发展。所有这些,预示着一切受压迫阶层的民主运动正在逐渐地向着同一的目标而汇合起来。运动的弱点,在于社会的基层分子还没有广泛地参加,非常重要的痛苦不堪的农民、工人、士兵与下层公教人员,还没有组织起来。运动的另一弱点,是参加运动的民主分子中,尚有许多人对于根据民主原则来转变时局,还缺乏明确的与坚决的精神。但是国际国内的时局,都迫着一切受压迫的阶层、党派、社会集团与个人,逐渐地觉悟与团结起来,执行自己的抗日、救国的神圣权利。不管国民党政府如何镇压,也不能阻止这一运动的发展。

国民党统治区内被压迫的一切阶层、党派、集团与个人的民主运动,应当有一个广大的发展,并把分散的力量逐渐统一起来,为着实现民族团结,建立联合政府,打败日本侵略者与建设新中国而斗争。为此目的,中国共产党与中国解放区,应当给予它们以一切可能的援助。

在国民党统治区,共产党人应当继续执行广泛的抗日民族统一战线政策,不管什么人,那怕昨天还是反对我们的,只要他今天不反对了,就应和他团结起来,为共同目标而奋斗。中国共产党人的一切工作,服从于动员与统一一切力量,彻底消灭日本侵略者与建设新中国这个总目标。

\subsection{中国沦陷区的任务}

在沦陷区,共产党人应当号召一切抗日人民,学习法国与意大利的榜样,将自己组织于各色团体中,组织地下军,准备武装起义,一俟时机成熟,配合从外部进攻的军队,里应外合地消灭日本侵略者。日本侵略者及其忠实走狗,对于我沦陷区内的兄弟姊妹们的无微不至的摧残、掠夺、奸淫与侮辱,激起了一切中国人的火一样的愤怒,报仇雪耻的时机快要到来了。沦陷区的人民,在东西战场及八路军新四军的胜利战争的鼓舞之下,极大地增高了他们的抗日情绪,他们迫切地需要组织起来,以便尽可能迅速地获得解放。因此,我们必须将沦陷区的工作提到和解放区的工作同等重要的地位上。必须有大批工作人员到沦陷区去工作。必须就沦陷区人民中训练与提拔大批的积极分子,参加当地的工作。在沦陷区中,东北四省沦陷最久,又是日本侵略者的产业中心与屯兵要地,我们应当加紧那里的地下工作。对于流亡的东北人民,应当加紧团结他们,准备收复失地。

在一切沦陷区,共产党人应当执行最广泛的抗日民族统一战线政策,不管什么人,只要是反对日本侵略者及其忠实走狗的,就要联合起来,为打倒共同敌人而斗争。

向一切帮助敌人反对同胞的伪军伪警及其它人员提出警告,是必要的。他们应该赶快认识自己的罪恶行为,及时回头,帮助同胞反对敌人,借以赎回自己的罪恶。否则敌人崩溃之日,民族纪律是不会对他们宽容的。

共产党人及抗日人民应当向一切有群众的伪组织进行争取说服工作,使他们站在反对民族敌人的战线上来。同时,对于那些罪大恶极不愿改悔的汉奸分子进行调查工作,以便在国土收复之日,能够迅速地依法治罪。

对于国民党内的反动分子组织汉奸反对中国人民、中国共产党、八路军、新四军及其他人民军队的背叛民族的罪恶行为,必须向他们提出警告,叫他们早日悔罪。否则,在国土收复之日,必然要将汉奸与汉奸组织者一体治罪,决不宽饶。

\subsection{中国解放区的任务}

中国人民在中国解放区的伟大斗争中,已经将我党的全部新民主主义的纲领变成了并正在变成被他们所热烈拥护与坚决执行的纲领,因而收获了显著的成绩,聚集了巨大的抗日力量;今后应当从各方面发展与巩固这种力量。

在目前条件下,中国解放区的军队应向一切被敌伪占领而又可能攻克的地方,不论是占领已久,或是新近占领的,发动广泛的进攻,借以扩大解放区,缩小沦陷区。

但是同时应当注意,敌人在目前还是强大的,它一定还要向解放区发动它的进攻,解放区军民必须随时准备粉碎敌人的进攻,并注意解放区的各项巩固工作。

应当争取在一切可能条件下扩大解放区的军队、游击队、民兵与自卫军,并加紧进行整训,增强其战斗力,为配合同盟国实行战略进攻,准备充分的军事力量。

在解放区,一方面,军队应实行拥政爱民的工作,另一方面,政府应领导人民实行拥军优抗的工作,使军民关系获得更大的改善。

共产党人在各个解放区的民选的三三制政府,即地方性联合政府的工作中,在社会工作中,应当继续过去的方针,和一切抗日民主分子,在新民主主义的共同纲领上,不分阶级、党派与信仰,进行很好的合作。

同样,在军事工作中,共产党人应当和一切愿意和我们合作的抗日民主分子,在解放区军队的内部和外部,很好地合作,为着消灭日本侵略者与保卫民主中国之伟大目的,而共同建设一个强大的人民军队。

为了提高工农劳动群众在抗日与生产力的积极性,适当的但是坚决的减租减息及改善工人与职员待遇的政策,应当继续执行。解放区的工作人员,必须努力学会做经济工作。必须动员一切可能的力量,大规模地发展解放区的农业、工业及贸易,改善军民生活,为坚持长期战争与消灭日本侵略者准备必要的物质条件。为此目的,必须实行劳动竞赛,奖励劳动英雄与模范工作者。在城市消灭日本侵略者以后,我们的工作人员,必须迅速学会做城市的经济工作。

为着提高解放区人民大众首先是广大的工人、农民、士兵群众的觉悟程度与培养大批工作干部,应当发展解放区的文化、教育事业。解放区的文化工作者与教育工作者在推进他们的工作时,应当根据目前的农村特点,根据农村人民的需要与自愿原则,采用适宜的内容与形式。

在推进解放区的各项工作时,必须十分爱惜当地的人力物力,任何地方均要作长期打算,避免滥用与浪费。这不但是为着打败日本人,而且是为着建设新中国。

在推进解放区的各项工作时,必须十分注意扶助本地人管理本地的事业,必须十分注意从本地人民优秀分子中大批地培养本地的工作干部。一切从外地去的人,如果不和本地人打成一片,如果不是满腔热情地勤勤恳恳地并适合情况地去帮助本地干部,爱惜他们,如同爱惜自己的兄弟姊妹一样,那就不能完成农村民主革命这个伟大的任务。

八路军、新四军及其他人民军队,每到一地,就应立即帮助本地人民,不但要组织以本地人民的干部为领导的民兵与自卫军,而且要组织以本地人民的干部为领导的地方部队与地方兵团。然后,就可以产生有本地人领导的主力部队与主力兵团。这是一项非常重要的任务。如果不能完成此项任务,就不能建立巩固的抗日根据地,也不能发展人民的军队。

当然,一切本地人,应当热烈地欢迎与帮助从外地去的人员与军队,借以完成共同任务。

关于对待暗藏的民族破坏分子问题,应当提起大家的注意,因为公开的敌人,公开的民族破坏分子,容易识别,也容易处置。暗藏的敌人,暗藏的民族破坏分子,则不容易识别,也就不容易处置。因此,对于这后一种人,既要采取严肃态度,又要采取谨慎态度。

根据信教自由的原则,中国解放区容许各派宗教存在。不论是基督教、天主教、回教、佛教及其它宗教,只要教徒们遵守政府法律,政府就给以保护。信教的和不信教的各有他们的自由,不许加以强迫或歧视。

最后,我们的大会应向各个解放区人民提议,尽可能迅速地在延安召开中国解放区人民代表会议,以便讨论统一各解放区的行动,加强各解放区的抗日工作,援助国民党统治区人民的抗日民主运动,援助沦陷区人民的地下军运动,促进全国人民的团结与联合政府的成立。中国解放区实际上现在已经成了全国广大人民所赖以抗日救国的重心,全国广大人民的希望寄托在我们身上,我们有责任不使他们失望。中国解放区人民代表会议的召集,将给中国人民的民族解放事业起一个巨大的推进作用。

\section{五 全党团结起来,为实现党的任务而斗争}

同志们,我们的任务是这样的巨大,我们的政策是这样的具体与明确,我们应该用怎样的工作态度去执行这些政策与实现这些任务呢?

目前国际国内的时局,在我们和中国人民面前,显示了光明的前途,具备了前所未有的有利条件,这是显然的,毫无疑义的。但是同时,依然存在着严重的困难条件。谁要是只看见光明一面,不看见困难一面,谁就会不能很好地为实现党的任务而斗争。

我们的党和中国人民一道,不论在整个党的二十四年历史中,在八年抗日战争中,为中国人民创造了巨大的力量,我们的工作成绩是很显然的,毫无疑义的。但是同时,我们的工作中依然存在着缺点。谁要是只看见成绩一面,不看见缺点一面,谁也就不会很好地为实现党的任务而斗争。

中国共产党自从它在一九二一年诞生以来,在其二十四年的历史中,经历了三次的伟大斗争,这就是北伐战争、土地革命和现在尚未完结的抗日战争。我们的党从它一开始,就是一个以马克思主义的理论为基础的党,这是因为这个主义是全世界无产阶级的最正确最革命的科学思想的结晶。马克思主义的普遍真理一经和中国革命的具体实践相结合,就使中国革命的面目为之一新,产生了新民主主义的整个历史阶段。在马克思主义的理论思想武装之下的中国共产党,在中国人民中产生了新的工作作风,这主要地就是理论与实践相结合的作风,和人民群众紧密地联系在一起的作风与自我批评的作风。

反映了全世界无产阶级实践斗争的马克思主义的普遍真理,只有在它和中国无产阶级及广大人民群众的革命斗争的具体实践相结合,才变为中国人民的有用武器。中国共产党正是这样做了。我们党的发展与进步,是从和一切违反这个真理的教条主义与经验主义作坚决斗争的过程中发展与进步起来的。教条主义脱离具体的实践,经验主义以局部经验误认为普遍真理,这两种机会主义的思想,都是违背马克思主义的。我们党在自己的二十四年奋斗中,克服了和正在克服着这些错误思想,使得我们的党在思想上极大地巩固了。我们党现在已有了一百二十一万党员,绝大多数是在抗日时期入党的,他们中存在着各种不纯正的思想,在抗日以前入党的党员中也有这种情形。几年来的整风工作收到了巨大的成效,使这些不纯正的思想受到了很大的纠正。今后应当继续这种工作,以“惩前毖后,治病救人”的精神,更大地展开党内的思想教育。要使各地各级党的领导骨干都懂得,理论与实践这样密切地相结合,是我们共产党人区别于其他任何政党的显著标志之一。因此,掌握思想教育,是团结全党进行伟大政治斗争的中心环节。如果这个任务不解决,党的一切政治任务是不能完成的。

我们共产党人区别于其它任何政党的又一个显著的标志,就是和最广大的人民群众取得最密切的联系。全心全意地为中国人民服务,一刻也不脱离群众;一切从人民的利益出发,而不是从自己小集团或自己个人的利益出发;向人民负责与向自己领导机关负责的一致性;这些就是我们的出发点。共产党人必须随时准备坚持真理,因为任何真理都是适合人民利益的。共产党人必须随时准备修正错误,因为任何错误都是不适合人民利益的。二十四年的经验告诉我们,凡属正确的任务、政策及工作作风,都是和当时当地的群众要求相适合,都是联系群众的。凡属错误了的任务、政策及工作作风,都是和当时当地的群众要求不相适合,都是脱离群众的。教条主义,经验主义,命令主义,尾巴主义,宗派主义,官僚主义,军阀主义,骄傲自大的工作态度等项弊病之所以一定不好,一定要不得,如果什么人有了这类弊病就一定要改正,就是因为它们脱离群众。我们的大会应该号召全党提起警觉,注意每一个工作环节上的每一个同志,不要让他脱离群众。教育每一个同志热爱人民群众,细心地倾听群众的呼声,每到一地就和那里的群众打成一片,不是高踞于群众之上,而是结合于群众之中,根据群众的觉悟程度,去启发与提高群众的觉悟,在群众出于内心自愿的原则之下,帮助群众逐步地组织起来,逐步地展开为当时当地内外环境所许可的一切必要的斗争。在一切工作中,命令主义是错误的,因为它超过群众的觉悟程度,违反了群众的自愿原则,害了急性病。我们的同志不要以为自己了解了的东西,广大群众也和自己一样一概都了解了。群众是否已经了解并且是否愿意行动起来,要到群众中去考察才会知道。如果我们这样做,就可以避免命令主义。在一切工作中,尾巴主义也是错误的,因为它落后于群众的觉悟程度,违反了领导群众前进一步的原则,害了慢性病。我们的同志不要以为自己尚不了解的东西,群众也一概不了解。许多时候,广大群众跑到我们的前头去了,迫切地需要前进一步了,我们的同志不能做广大群众的领导者,却反映了一部分落后分子的意见,并将此种落后分子的意见误认为广大群众的意见,做了落后分子的尾巴。总之,应使每个同志明了,共产党人的一切言论、行动,以是否合乎最广大人民群众的最大利益,是否为最广大人民群众所拥护为最高标准。应使每一个同志懂得,只要我们依靠人民,坚决地相信人民群众的创造力是无穷无尽的,因而信任人民,和人民打成一片,那就任何困难也能克服,任何敌人也不能压倒我们,而只会被我们所压倒。

有无认真的自我批评,也是我们和其他政党互相区别的显著标志之一。我们曾经说过,房子是应该经常打扫的,不打扫就会积满了灰尘。脸是应该经常洗的,不洗也就会灰尘满面。我们同志的思想,我们党的工作,也会发生灰尘的,也应该打扫与洗涤。“流水不腐,户枢不蟗”,是说它们在不停的运动中抵抗了微生物或其它生物的侵蚀。对于我们,经常地检讨工作,在检讨中推广民主作风,不惧怕批评与自我批评,实行“知无不言,言无不尽”,“言者无罪,闻者足戒”,“有则改之,无则加勉”,这些中国人民的有益的格言,正是抵抗错误、缺点这类政治微生物侵蚀我们同志的思想与我们党的机体的唯一有效的方法。以“惩前毖后、治病救人”为宗旨的整风运动之所以发生了很大的效力,就是因为我们在这个运动中展开了正确的而不是歪曲的、认真的而不是敷衍的批评与自我批评。以中国最广大人民的最大利益为出发点的中国共产党人,相信自己的事业是完全合乎正义的,不惜牺牲自己个人的一切,随时准备拿出自己的生命去殉我们的事业,难道我们还有什么错误的不适合人民需要的思想,观点,意见,办法,舍不得丢掉的吗?难道我们还欢迎任何政治的灰尘,政治的微生物来点污我们的清洁的面貌与侵蚀我们的健全的机体吗?无数革命先烈为了人民的利益牺牲了他们的生命,使我们每个活着的人想起他们就心里难过,难道我们还有什么个人利益或错误、缺点,不能牺牲吗?

同志们,我们的大会闭幕之后,我们就要上战场去,根据大会的决议,为着打倒日本侵略者与建设新中国而奋斗。为达此目的,我们要同全国人民团结起来。我重说一遍,不管什么阶级,什么政党,什么社会集团或个人,只要他是赞成打倒日本侵略者与建设新中国的,我们就要和他团结起来。为达此目的,我们要把我们的党在民主集中制的组织与纪律之下,比较过去还要好还要坚强地团结起来,不论什么同志,只要他是愿意服从党纲、党章和党的决议的,我们就要和他团结起来。我们的党,在北伐战争时期,不超过五万党员,并且后来大部分被当时的敌人打散了。在土地革命时期,不超过三十万党员,后来大部分也被当时的敌人打散了。现在我们有了一百二十余万党员,这一回无论如何不要被敌人打散。只要我们能吸取三个时期的经验,采取谦虚态度,防止骄傲态度,在党内,和全体同志更好地团结起来,在党外,和全国人民更好地团结起来,就可以保证,不但不会被敌人打散,相反地,一定要把日本侵略者及其忠实走狗坚决、彻底、干净、全部消灭之,并且在消灭他们之后,把一个独立、自由、民主、统一与富强的中国建设起来。

三次革命的经验,尤其是抗日战争的经验,给了我们及中国人民这样一种信心:没有中国共产党的努力,没有中国共产党人做中国人民的中流砥柱,中国的独立、自由、民主、统一是不可能的,中国的工业化和农业近代化也是不可能的。

同志们,有了三次革命经验的中国共产党,我坚决相信,我们是能够完成我们的伟大政治任务的。

成千成万的人民的与党的先烈,为着人民的利益,在我们的前头英勇地牺牲了,让我们高举起他们的旗帜,踏着他们的血迹前进吧!

一个独立、自由、民主、统一与富强的中国不久就要诞生了,让我们迎接这个伟大的日子吧!

打倒日本侵略者!

中国人民解放万岁!
